%
% @author  Andreas Febrian
% @version 2.2.1
% @edit by Ichlasul Affan
%
% Mendaftar seluruh istilah yang mungkin akan perlu dijadikan
% italic atau bold pada setiap kemunculannya dalam dokumen.
%
% v2.2.1 - support untuk Glossary (daftar istilah)
%

% Istilah/alias yang tidak perlu dimasukkan ke dalam Glossary/Daftar Istiilah
\var{\license}{\f{Creative Common License 1.0 Generic}}
\var{\bslash}{$\setminus$}


\makeglossaries

% Contoh istilah
\newglossaryentry{latex}
{
	name=\LaTeX,
	description={Sebuah \f{mark up language} yang didesain khusus untuk karya tulis ilmiah}
}

\newglossaryentry{aristotle}
{
	name={Aristotle},
	description={Sebuah \f{framework} yang didesain untuk menyelesaikan permasalahan \f{logical inference} menggunakan LLM.}
}

% Contoh Akronim
\newacronym{pdf}{PDF}{\f{Portable Document Format}}
\newacronym{llm}{LLM}{\f{Large Language Model}}
\newacronym{slm}{SLM}{\f{Small Language Model}}
\newacronym{fol}{FOL}{\f{First Order Logic}}
\newacronym{pnf}{PNF}{\f{Prenex Normal Form}}
\newacronym{cnf}{CNF}{\f{Conjunctive Normal Form}}
\newacronym{nlp}{NLP}{\f{Natural Language Processing}}
