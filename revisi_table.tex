\documentclass[a4paper, 12pt]{article}

% Required Packages
\usepackage[left=2.5cm, right=2.5cm, top=2.5cm, bottom=2.5cm]{geometry} % Adjust margins
\usepackage{tabularx} % For adjustable column widths (X columns)
\usepackage{float}    % For [H] placement
\usepackage{multirow} % For multi-row cells if needed
\usepackage{array}    % Helper for table formatting

\begin{document}

% Title for the document (Optional)
\begin{center}
    \Large \textbf{Log Revisi Dokumen}
\end{center}
\vspace{1cm}

% The Table
\begin{table}[H]
    \centering
    \caption{Daftar Revisi dan Perubahan}
    \label{tab:revisi_log}
    \renewcommand{\arraystretch}{1.5} % Increases row height for readability

    % Column setup: 
    % c = center (small width)
    % X = auto-width (wraps long text)
    % X = auto-width (wraps long text)
    % l = left align (adjust to 'c' if you prefer centered status)
    \begin{tabularx}{\textwidth}{|c|X|X|l|}
        \hline
        \textbf{No}                                                                                                   & \centering\textbf{Masukan / Isu Awal} & \centering\textbf{Perubahan / Tindak Lanjut} & \textbf{Hal/Bab} \\
        \hline

        % Example Row 1
        1                                                                                                             &
        Penjelasan mengenai metodologi dianggap kurang detail, khususnya pada bagian alur data.                       &
        Menambahkan diagram alir baru dan memperjelas deskripsi langkah-langkah preprocessing data.                   &
        Bab 3                                                                                                                                                                                                                   \\
        \hline

        % Example Row 2
        2                                                                                                             &
        Terdapat kesalahan penulisan (typo) pada istilah asing yang tidak dicetak miring.                             &
        Semua istilah asing (e.g., \textit{framework}, \textit{backend}) telah diubah menjadi format \textit{italic}. &
        All                                                                                                                                                                                                                     \\
        \hline

        % Example Row 3
        3                                                                                                             &
        Rumus 2.1 tidak konsisten dengan referensi yang dikutip.                                                      &
        Memperbaiki notasi pada Rumus 2.1 agar sesuai dengan Referensi [5].                                           &
        Hal. 12                                                                                                                                                                                                                 \\
        \hline
    \end{tabularx}
\end{table}

\end{document}