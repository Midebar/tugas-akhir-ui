\documentclass[a4paper, 12pt]{article}

\usepackage[left=2.5cm, right=2.5cm, top=2.5cm, bottom=2.5cm]{geometry}
\usepackage{tabularx}
\usepackage{float}
\usepackage{multirow}
\usepackage{array}

\begin{document}

\begin{center}
    \Large \textbf{Log Revisi Dokumen}
\end{center}
\vspace{1cm}

\begin{table}[H]
    \centering
    \caption{Daftar Revisi dan Perubahan}
    \label{tab:revisi_log}
    \renewcommand{\arraystretch}{1.5} % Increases row height for readability
    
    % Column setup: 
    % c = center (small width)
    % X = auto-width (wraps long text)
    % X = auto-width (wraps long text)
    % l = left align (adjust to 'c' if prefer centered status)
    \begin{tabularx}{\textwidth}{|c|X|X|l|}
        \hline
        \textbf{No}                                                                                                                                                                                                                               & \centering\textbf{Masukan / Isu Awal} & \centering\textbf{Perubahan / Tindak Lanjut} & \textbf{Hal/Bab} \\
        \hline
        
        1                                                                                                                                                                                                                                         & 
        Ilustrasi tahapan dari input awal hingga hasil akhir                                                                                                                                                                                      & 
        Menambahkan ilustrasi input dari awal                                                                                                                                                                                                     & 
        Bagian 3.4                                                                                                                                                                                                                                                                                                                                          \\
        \hline
        
        2                                                                                                                                                                                                                                         & 
        Membutuhkan penjelasan pada framework Aristotle pada 4 modul utama                                                                                                                                                                        & 
        Memperjelas 4 modul utama / tahapan utama dan penggabungan ke search and resolve                                                                                                                                                          & 
        Bagian 3.2                                                                                                                                                                                                                                                                                                                                          \\
        \hline
    \end{tabularx}
\end{table}

\end{document}