%-------------------------%
\pagestyle{onlypage}
\chapter*{\kataPengantar}
%-----------------------------------------------------------------------------%
Puji dan syukur Penulis panjatkan kepada Tuhan Yang Maha Esa atas segala berkat, rahmat, dan karunia-Nya, sehingga Penulis dapat menyelesaikan skripsi dengan judul \Judul \, ini dengan baik. Penulisan skripsi ini dilakukan dalam rangka memenuhi salah satu syarat untuk mencapai gelar Sarjana pada Program Studi Ilmu Komputer, Fakultas Ilmu Komputer, Universitas Indonesia. Oleh karena itu, pada kesempatan ini Penulis ingin menyampaikan rasa terima kasih dan penghargaan yang setinggi-tingginya kepada:

\begin{enumerate}[topsep=0pt,itemsep=-1ex,partopsep=1ex,parsep=1ex]
	\item Pak Ari, selaku dosen pembimbing satu yang telah meluangkan waktu, tenaga, dan pikiran untuk memberikan arahan, diskusi mendalam mengenai \textit{Neuro-Symbolic}, serta motivasi kepada Penulis selama proses penelitian ini.
	\item Seluruh Dosen Fakultas Ilmu Komputer yang telah memberikan bekal ilmu pengetahuan yang sangat bermanfaat bagi Penulis selama masa perkuliahan.
	\item Orang tua tercinta, yang senantiasa memberikan doa, kasih sayang, dukungan moral, dan material yang tak terhingga.
	\item Teman-teman seperjuangan yang telah menjadi teman diskusi dan berbagi cerita selama masa perkuliahan.
	\item Pihak pengembang \textit{open-source} (komunitas AI Singapore, Qwen Team, dan GoTo Group) yang telah menyediakan model SEA-LION, Qwen, dan Sahabat-AI secara terbuka, yang memungkinkan penelitian ini terlaksana dengan sumber daya terbatas.
	\item Platform \textit{cloud computing} yang menyediakan sumber daya komputasi untuk menggunakan model pembelajaran mesin, yaitu Runpod.
	\item Semua referensi dan literatur yang telah membantu dalam penelitian dan penulisan skripsi ini.
\end{enumerate}

Tugas akhir ini disusun dengan bantuan teknologi kecerdasan artifial dengan platform \textit{GeminiAI} untuk membantu dalam pemilihan kata dan perbaikan tata bahasa, terutama pada bagian landasan teori.
Penulis menyadari bahwa laporan \type~ini masih jauh dari sempurna, mengingat keterbatasan pengetahuan dan pengalaman Penulis dalam bidang \textit{Natural Language Processing} (NLP) yang terus berkembang. Oleh karena itu, apabila terdapat kesalahan atau kekurangan dalam laporan ini, Penulis memohon agar kritik dan saran yang membangun bisa disampaikan langsung melalui \f{e-mail} \code{mikhael.deo@ui.ac.id}.

Akhir kata, Penulis berharap semoga skripsi ini dapat memberikan manfaat bagi pengembangan ilmu pengetahuan, khususnya dalam bidang penalaran mesin pada Bahasa Indonesia.



% \begin{figure}[h] % Added [h] for standard latex placement

% \begin{figure}
% 	\centering
% 	\includegraphics[width=0.65\textwidth]
% 	{assets/pics/creative_commons.png}
% 	\caption*{\license}
% 	\label{fig:lisensi}
% \end{figure}

% Terkait template ini, gambar lisensi di atas diambil dari \url{http://creativecommons.org/licenses/by-nc-sa/1.0/deed.en_CA}. Jika ingin mengentahui lebih lengkap mengenai \license, silahkan buka \url{http://creativecommons.org/licenses/by-nc-sa/1.0/legalcode}.
% Seluruh dokumen yang dibuat dengan menggunakan template ini sepenuhnya menjadi hak milik pembuat dokumen dan bebas didistribusikan sesuai dengan keperluan masing-masing.
% Lisensi hanya berlaku jika ada orang yang membuat template baru dengan menggunakan template ini sebagai dasarnya.

% Penyusun template ingin berterima kasih kepada Andreas Febrian, Lia Sadita, Fahrurrozi Rahman, Andre Tampubolon, dan Erik Dominikus atas kontribusinya dalam template yang menjadi pendahulu template ini.
% Penyusun template juga ingin mengucapkan terima kasih kepada Azhar Kurnia atas kontribusinya dalam template yang menjadi pendahulu template ini.

% Semoga template ini dapat membantu orang-orang yang ingin mencoba menggunakan \latex.
% Semoga template ini juga tidak berhenti disini dengan ada kontribusi dari para penggunanya.
% Jika Anda memiliki perubahan yang dirasa penting untuk disertakan dalam template, silakan lakukan \f{fork} repositori Git template ini di \url{https://gitlab.com/ichlaffterlalu/latex-skripsi-ui-2017}, lalu lakukan \f{merge request} perubahan Anda terhadap \f{branch} \code{master}.
% Kami berharap agar \f{template} ini dapat terus diperbarui mengikuti perubahan ketentuan dari pihak Rektorat Universitas Indonesia, dan hal itu tidak mungkin terjadi tanpa kontribusi dari teman-teman sekalian.

% Untuk input gambar tanda tangan, silahkan sesuaikan xshift, yshift, dan width dengan gambar tanda tangan Anda
%\begin{tikzpicture}[remember picture,overlay,shift={(current page.north east)}]
%\node[anchor=north east,xshift=-3cm,yshift=-6.2cm]{\includegraphics[width=3cm]{assets/pics/tanda_tangan_wikipedia.png}};
%\end{tikzpicture}

\vspace*{0.1cm}
\begin{flushright}
	Depok, \tanggalSiapSidang\\[0.1cm]
	\ifx\blank\npmDua
		\vspace*{1.5cm}
		\penulisSatu
	\else
		Tim Penulis
	\fi
	
\end{flushright}
