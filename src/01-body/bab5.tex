%-----------------------------------------------------------------------------%
\chapter{\babLima}
\label{bab:5}
%-----------------------------------------------------------------------------%
TTTTTTTTTTTTTTTTT
Awalnya, \f{template} ini hanya digunakan untuk Tugas Akhir (Skripsi) mahasiswa S1 di Fakultas Ilmu Komputer, Universitas Indonesia. Seiring berkembangnya kegiatan pendidikan dan kemahasiswaan di lingkup Fakultas Ilmu Komputer hingga tingkat universitas, penyusun \f{template} menyadari ada kasus-kasus lain yang bisa menggunakan format Tugas Akhir UI. Beberapa di antaranya adalah tesis S2, disertasi S3, dan laporan kegiatan/kerja praktik. Oleh karena itu, perlu ada penjelasan terkait berbagai kasus penggunaan (\f{use case}) untuk \f{template} \gls{latex} ini, dan bagaimana cara pengguna bisa memanfaatkan \f{template} untuk kasus tersebut.
\todo{Sejatinya bab ini digunakan untuk membahas inti penelitian Anda. Bab lima pada tugas akhir S1 umumnya merupakan pembahasan analisis dari penelitian. Namun, sekali lagi, sesuaikan dengan kebutuhan Anda. Tesis atau disertasi tentunya berbeda dengan skripsi.}


%-----------------------------------------------------------------------------%
\section{Tugas Akhir Individu S1, Proposal Tesis, dan Tesis S2}
\label{sec:skripsiIndividu}
%-----------------------------------------------------------------------------%
Tugas Akhir Individu di Fakultas Ilmu Komputer Universitas Indonesia berlaku sama dengan Tugas Akhir atau Skripsi mahasiswa S1 di fakultas lain di Universitas Indonesia.
Proposal Tesis dan Tesis (di beberapa jurusan disebut Karya Akhir) di Fakultas Ilmu Komputer Universitas Indonesia juga berlaku sama dengan Tesis mahasiswa S2 di fakultas lain di Universitas Indonesia.
Format yang digunakan untuk semua fakultas juga sama, mengacu ke Keputusan Rektor Universitas Indonesia nomor 2143/SK/R/UI/2017 tentang Pedoman Teknis Penulisan Tugas Akhir Mahasiswa Universitas Indonesia.
Sejak versi 2.0.0, \f{template} ini sudah mengacu ke Keputusan Rektor UI tersebut.
Pada versi tersebut juga dukungan untuk cetak skripsi atau tesis bolak-balik sudah tersedia.
Tidak ada perubahan khusus yang perlu dilakukan terhadap konfigurasi \f{template} untuk Tugas Akhir untuk Mahasiswa S1 atau Proposal Tesis dan Tesis untuk Mahasiswa S2.
Anda bisa mengikuti tahapan berikut untuk memulai penulisan Anda:
\begin{enumerate}
	\item Buka \code{config/settings.tex}. Terdapat lima bagian yang perlu dilengkapi:
	      \begin{itemize}
		      \item \bo{Judul dokumen}: Anda bisa memasukkan judul dalam bahasa Indonesia dan bahasa Inggris di sini.
		      \item \bo{Tipe dokumen}: Pada variabel \code{\bslash{}type}, cukup tuliskan "Skripsi" atau "Tugas Akhir", sesuaikan dengan aturan dari Fakultas masing-masing.
		            Isi variabel \code{\bslash{}jenjang} dengan "Sarjana" atau "Magister". Kosongkan variabel lainnya yang tidak relevan (jangan dihapus).
		      \item \bo{Informasi penulis}: Karena pada kasus ini, tugas akhir Anda bersifat individu, cukup isi variabel \code{\bslash{}penulisSatu} dengan nama Anda, \code{\bslash{}npmSatu} dengan NPM Anda, \code{\bslash{}programSatu} dengan nama program studi Anda dalam bahasa Indonesia, dan \code{\bslash{}studyProgramSatu} dengan nama program studi Anda dalam bahasa Inggris.
		            Untuk variabel lain mohon agar tetap dikosongkan (namun jangan dihapus) sehingga \f{template} bisa mendeteksi bahwa Anda akan menuliskan skripsi individu.
		      \item \bo{Informasi dosen pembimbing dan penguji}: Pada umumnya, dosen pembimbing skripsi di UI terdiri dari satu atau dua orang dosen, dan penguji skripsi di UI terdiri dari dua orang dosen.
		            Silakan isi variabel yang relevan dan kosongkan variabel lainnya (namun jangan dihapus).
		      \item \bo{Informasi lain}: Anda bisa melihat komentar di setiap variabel untuk mengetahui apa yang harus diisi di setiap variabel.
		      \item \bo{Judul setiap bab}: Silakan isi variabel yang ada untuk judul setiap bab. Jika ada bab yang ingin ditambahkan sebelum bab kesimpulan (misal: bab 6, bab 7), Anda dapat membuat variabel baru, contohnya: \code{\bslash{}Var\{\bslash{}bab6\}\{Analisis Pendapat Pengguna Aplikasi\}}.
		      \item Bagian lainnya seperti "Capitalized Variables" tidak perlu dimodifikasi. Variabel-variabel tersebut menunjang fungsi-fungsi khusus di \f{template}, salah satunya adalah versi \f{all caps} dari judul skripsi di halaman judul.
	      \end{itemize}
	\item Setelah mengisi konfigurasi, Anda bisa periksa halaman-halaman awal dokumen.
	      Jika terdapat ketidaksesuaian pada ukuran atau jarak antar elemen, Anda bisa mengatur melalui berkas-berkas yang ada di \code{src/00-frontMatter}.
	      Halaman pengesahan sidang yang dipakai di format Tugas Akhir Individu ada di \code{src/00-frontMatter/pengesahanSidang.tex}.
	      Silakan perbesar atau perkecil ukuran yang ada pada kode \code{\bslash{}vspace*\{...\}}, untuk menyesuaikan \f{spacing}.
	      Tahapan ini akan berguna terutama jika judul tugas akhir Anda cukup panjang sehingga beberapa teks ada yang terlempar ke halaman berikutnya.
	      Jika ada perubahan kode yang signifikan, Anda bisa mengusulkan ke penyusun \f{template}.
	      Keterangan lebih lanjut terkait cara kontribusi dapat dilihat di berkas \code{README.md} dan \code{CONTRIBUTING}.
	\item Anda juga bisa mengatur beberapa hal sebagai berikut:
	      \begin{itemize}
		      \item Pelajari cara sitasi dengan melihat \sect~\ref{sec:bibtex} dan cara melakukan \f{cross-reference} dengan melihat \sect~\ref{sec:crossReference}.
		            Kedua fitur tersebut merupakan fitur yang sangat penting dalam penulisan skripsi menggunakan \gls{latex}.
		      \item Jika fakultas Anda memerlukan format sitasi selain APA (yang menjadi \f{default} di tingkat universitas), silakan baca \sect~\ref{sec:bibtexChangeFormat}.
		      \item Jika Anda membutuhkan support untuk selain tulisan alfabet, silakan baca \sect~\ref{sec:multilanguageSupport}.
		            Jika Anda membutuhkan penulisan notasi matematis, silakan baca \sect~\ref{sec:mathEqu}.
		            Jika Anda membutuhkan penulisan kode program, silakan baca \sect~\ref{sec:codeListing}.
	      \end{itemize}
	\item Di akhir penulisan, Anda perlu memeriksa ulang tulisan Anda secara lebih teliti untuk memaksimalkan penggunaan kertas, sebisa mungkin hindari \f{unused space}. Selain itu, perhatikan juga pemenggalan yang dilakukan \gls{latex} apakah sudah sesuai atau belum. Jika ada pemenggalan yang kurang sesuai, silakan tambahkan di \code{\_internals/hypeindonesia.tex} dan \f{request} untuk kontribusi.
	      Keterangan lebih lanjut terkait cara kontribusi dapat dilihat di berkas \code{README.md} dan \code{CONTRIBUTING}.
\end{enumerate}

%-----------------------------------------------------------------------------%
\section{Tugas Akhir Kelompok S1}
\label{sec:skripsiKelompok}
%-----------------------------------------------------------------------------%
Beberapa fakultas, salah satunya Fakultas Ilmu Komputer Universitas Indonesia (sejak tahun 2022) mengizinkan pengerjaan skripsi secara berkelompok paling banyak 3 (tiga) orang.
Format yang digunakan juga mengacu ke Keputusan Rektor Universitas Indonesia nomor 2143/SK/R/UI/2017 tentang Pedoman Teknis Penulisan Tugas Akhir Mahasiswa Universitas Indonesia, namun ada penyesuaian di beberapa hal.
Sejak versi 2.1.3, \f{template} ini mendukung \f{format} Tugas Akhir kelompok dengan menyesuaikan bagian depan dari \f{template}.
Untuk memanfaatkan \f{format} tersebut, silakan ikuti tahapan berikut:
\begin{enumerate}
	\item Buka \code{config/settings.tex}. Isi variabel pada bagian "\bo{Informasi Penulis}" untuk penulis pertama, kedua dan ketiga secara berurutan. Misal: \code{\bslash{}penulisSatu} untuk nama penulis pertama, \code{\bslash{}penulisDua} untuk nama penulis kedua, dan \code{\bslash{}penulisTiga} untuk nama penulis ketiga.
	      Pastikan Anda mengisi data secara lengkap pada variabel yang sesuai.
	      Jika kelompok Anda hanya terdiri dari 2 (dua) orang, maka variabel-variabel data penulis ketiga harus dikosongkan (namun jangan dihapus).
	      \f{Template} akan menyesuaikan \f{format} sesuai dengan jumlah anggota kelompok di skripsi Anda.
	\item Setelah mengisi konfigurasi, Anda bisa periksa halaman-halaman awal dokumen.
	      Jika terdapat ketidaksesuaian pada ukuran atau jarak antar elemen, Anda bisa mengatur melalui berkas-berkas yang ada di \code{src/00-frontMatter}.
	      Halaman pengesahan sidang yang dipakai di format Tugas Akhir Kelompok ada di \code{src/00-frontMatter/pengesahanSidang.tex}.
	      Silakan perbesar atau perkecil ukuran yang ada pada kode \code{\bslash{}vspace*\{...\}}, untuk menyesuaikan \f{spacing}.
	      Tahapan ini akan berguna terutama jika judul tugas akhir Anda dan data kelompok Anda cukup panjang sehingga beberapa teks ada yang terlempar ke halaman berikutnya.
	      Jika ada perubahan kode yang signifikan, Anda bisa mengusulkan ke penyusun \f{template}.
	      Keterangan lebih lanjut terkait cara kontribusi dapat dilihat di berkas \code{README.md} dan \code{CONTRIBUTING}.
\end{enumerate}


%-----------------------------------------------------------------------------%
\section{Laporan Ilmiah dan Disertasi S3}
\label{sec:disertasi}
%-----------------------------------------------------------------------------%
Disertasi S3 dan laporan-laporan lain yang diwajibkan untuk jenjang S3 juga menggunakan format sesuai Keputusan Rektor Universitas Indonesia nomor 2143/SK/R/UI/2017 tentang Pedoman Teknis Penulisan Tugas Akhir Mahasiswa Universitas Indonesia, namun ada penyesuaian di beberapa hal.
Salah satu penyesuaian yang perlu dilakukan adalah istilah pembimbing yang berganti menjadi Promotor, Kopromotor.
Jumlah penguji juga lebih banyak, bisa mencapai 6 orang dosen penguji.
Sejak versi 2.1.2, \f{template} ini mendukung \f{format} disertasi dengan menyesuaikan bagian depan dari \f{template}.
Untuk memanfaatkan \f{format} tersebut, silakan ikuti tahapan berikut:
\begin{enumerate}
	\item Buka \code{config/settings.tex}.
	      \begin{itemize}
		      \item Pada bagian "\bo{Tipe Dokumen}", variabel \code{\bslash{}type} bisa diisi dengan "Disertasi" atau tipe dokumen lainnya.
		            Variabel \code{\bslash{}jenjang} wajib diisi dengan "Doktor".
		      \item Pada bagian "\bo{Informasi Pembimbing dan Penguji}", isi nama lengkap dan gelar Promotor pada variabel \code{\bslash{}pembimbingSatu}, dan Kopromotor pada variabel  \code{\bslash{}pembimbingDua} (jika kopromotor ada dua orang, variabel  \code{\bslash{}pembimbingTiga} bisa diisi).
		            Untuk penguji, Anda bisa mengisi secara berurutan dari  \code{\bslash{}pengujiSatu} hingga \code{\bslash{}pengujiEnam}.
	      \end{itemize}
	      Konfigurasi untuk dokumen laporan ilmiah S3 tidak mendukung format Tugas Akhir Kelompok.
	\item Setelah mengisi konfigurasi, Anda bisa periksa halaman-halaman awal dokumen.
	      Jika terdapat ketidaksesuaian pada ukuran atau jarak antar elemen, Anda bisa mengatur melalui berkas-berkas yang ada di \code{src/00-frontMatter}.
	      Halaman pengesahan sidang yang dipakai di format laporan ilmiah S3 ada di \code{src/00-frontMatter/pengesahanSidangS3.tex}.
	      Jika "Halaman Pengesahan" menjadi dua halaman, hal tersebut adalah lumrah.
	      Jika ada hal yang tidak lumrah, silakan perbesar atau perkecil ukuran yang ada pada kode \code{\bslash{}vspace*\{...\}}, untuk menyesuaikan \f{spacing}.
	      Jika ada perubahan kode yang signifikan, Anda bisa mengusulkan ke penyusun \f{template}.
	      Keterangan lebih lanjut terkait cara kontribusi dapat dilihat di berkas \code{README.md} dan \code{CONTRIBUTING}.
\end{enumerate}


%-----------------------------------------------------------------------------%
\section{Laporan Kerja Praktik}
\label{sec:laporanKerjaPraktik}
%-----------------------------------------------------------------------------%
Mata kuliah Kerja Praktik umumnya ditawarkan bagi individu sebagai mata kuliah bernilai SKS untuk mempresentasikan dan mendokumentasikan pekerjaan magang di industri melalui laporan karya ilmiah.
Laporan Kerja Praktik di Fakultas Ilmu Komputer UI (dan sebagian fakultas yang menyediakan mata kuliah Kerja Praktik) juga menggunakan format sesuai Keputusan Rektor Universitas Indonesia nomor 2143/SK/R/UI/2017 tentang Pedoman Teknis Penulisan Tugas Akhir Mahasiswa Universitas Indonesia, namun ada penyesuaian di beberapa hal.
Salah satu penyesuaian yang perlu dilakukan adalah halaman persetujuan yang berbeda karena Kerja Praktik tidak memerlukan sidang.
Selain itu, ada beberapa halaman yang tidak diperlukan seperti Pernyataan Orisinalitas dan Persetujuan Publikasi.
Sejak versi 2.1.2, \f{template} ini mendukung \f{format} laporan kerja praktik dengan menyesuaikan bagian depan dari \f{template}.
Untuk memanfaatkan \f{format} tersebut, silakan ikuti tahapan berikut:
\begin{enumerate}
	\item Buka \code{config/settings.tex}.
	      \begin{itemize}
		      \item Pada bagian "\bo{Tipe Dokumen}", variabel \code{\bslash{}type} wajib diisi dengan "Laporan Kerja Praktik".
		            Variabel \code{\bslash{}jenjang} wajib diisi dengan "Sarjana".
		      \item Pada bagian "\bo{Informasi Pembimbing dan Penguji}", isi nama lengkap dan gelar dosen kelas Kerja Praktik pada \code{\bslash{}pembimbingSatu}, dan kosongkan semua variabel lain pada bagian tersebut (namun jangan dihapus).
	      \end{itemize}
	      Konfigurasi untuk Laporan Kerja Praktik tidak mendukung format Tugas Akhir Kelompok.
	\item Setelah mengisi konfigurasi, Anda bisa periksa halaman-halaman awal dokumen. Jika terdapat ketidaksesuaian pada ukuran atau jarak antar elemen, Anda bisa mengatur melalui berkas-berkas yang ada di \code{src/00-frontMatter}.
	      Halaman persetujuan yang dipakai di format Laporan Kerja Praktik ada di \code{src/00-frontMatter/pengesahanKP.tex}.
	      Silakan perbesar atau perkecil ukuran yang ada pada kode \code{\bslash{}vspace*\{...\}}, untuk menyesuaikan \f{spacing}.
	      Jika ada perubahan kode yang signifikan, Anda bisa mengusulkan ke penyusun \f{template}.
	      Keterangan lebih lanjut terkait cara kontribusi dapat dilihat di berkas \code{README.md} dan \code{CONTRIBUTING}.
\end{enumerate}


%-----------------------------------------------------------------------------%
\section{Laporan Kegiatan Merdeka Belajar Kampus Merdeka}
\label{sec:laporanKampusMerdeka}
%-----------------------------------------------------------------------------%
Program Merdeka Belajar Kampus Merdeka\footnote{\url{https://kampusmerdeka.kemdikbud.go.id/}} merupakan program \f{flagship} dari Kementerian Pendidikan, Kebudayaan, Riset, dan Teknologi (Kemendikbud) Republik Indonesia yang bertujuan untuk memberikan peluang mahasiswa mendapatkan pengalaman belajar di luar kampus.
Terdapat banyak pilihan program Kampus Merdeka yang tersedia bagi mahasiswa UI, beberapa di antaranya adalah Magang Bersertifikat, Studi Independen Bersertifikat (termasuk Program Bangkit\footnote{\url{https://www.dicoding.com/programs/bangkit}}), dan beberapa program lain di tingkat UI seperti \f{Build Your Own Course} (BYOC).
Pada akhir program, mahasiswa diminta menyusun laporan dengan format yang disediakan untuk Kemendikbud, yang tentunya hanya tersedia untuk program dan jalur yang dikelola Kemendikbud.
Beberapa program seperti BYOC dan jalur yang diselenggarakan UI seperti Kampus Merdeka Mandiri tidak memiliki akses ke template Kemendikbud.
Di Fakultas Ilmu Komputer, laporan MBKM yang tidak melewati jalur yang dikelola Kemendikbud menggunakan laporan akhir layaknya Laporan Kerja Praktik yang formatnya menggunakan aturan Keputusan Rektor Universitas Indonesia nomor 2143/SK/R/UI/2017 tentang Pedoman Teknis Penulisan Tugas Akhir Mahasiswa Universitas Indonesia, dengan beberapa penyesuaian.
Salah satu penyesuaian yang perlu dilakukan adalah halaman persetujuan yang berbeda karena Kampus Merdeka tidak memerlukan sidang, namun berbeda dengan Kerja Praktik, laporan kegiatan Kampus Merdeka membutuhkan persetujuan dari mitra.
Sejak versi 2.1.3, \f{template} ini mendukung \f{format} laporan kerja praktik dengan menyesuaikan bagian depan dari \f{template}.
Untuk memanfaatkan \f{format} tersebut, silakan ikuti tahapan berikut:
\begin{enumerate}
	\item Buka \code{config/settings.tex}.
	      \begin{itemize}
		      \item Pada bagian "\bo{Tipe Dokumen}", variabel \code{\bslash{}type} wajib diisi dengan "Kampus Merdeka".
		            Variabel \code{\bslash{}jenjang} wajib diisi dengan "Sarjana".
		            Variabel \code{\bslash{}kampusMerdekaType} wajib diisi dengan tipe kegiatan atau jalur yang diambil, misal: Magang, Studi Independen, Bangkit, dsb.
		            Jika program memiliki mitra, variabel \code{\bslash{}partnerPosition} wajib diisi dengan jabatan yang dimiliki perwakilan mitra yang akan menandatangani laporan Anda.
		            Jika program memiliki mitra, variabel \code{\bslash{}partnerInstance} wajib diisi dengan instansi, perusahaan, atau program yang menjadi tempat kerja perwakilan mitra yang akan menandatangani laporan Anda.
		      \item Pada bagian "\bo{Informasi Pembimbing dan Penguji}", isi nama lengkap dan gelar dosen penanggungjawab program Kampus Merdeka yang diambil (untuk mahasiswa Fasilkom UI) atau Pembimbing Akademik (untuk fakultas lain) pada \code{\bslash{}pembimbingSatu}.
		            Kemudian, isi nama lengkap perwakilan penyelia atau manajer dari mitra tempat kegiatan pada \code{\bslash{}pembimbingDua}.
		            Jika program tidak memiliki mitra (misalkan BYOC), kosongkan variabel \code{\bslash{}pembimbingDua}.
		            Kosongkan semua variabel lain pada bagian tersebut (namun jangan dihapus).
	      \end{itemize}
	      Konfigurasi untuk Kampus Merdeka tidak mendukung format Tugas Akhir Kelompok.
	\item Setelah mengisi konfigurasi, Anda bisa periksa halaman-halaman awal dokumen. Jika terdapat ketidaksesuaian pada ukuran atau jarak antar elemen, Anda bisa mengatur melalui berkas-berkas yang ada di \code{src/00-frontMatter}.
	      Halaman persetujuan yang dipakai di format Laporan Kerja Praktik ada di \code{src/00-frontMatter/pengesahanMBKM.tex}.
	      Silakan perbesar atau perkecil ukuran yang ada pada kode \code{\bslash{}vspace*\{...\}}, untuk menyesuaikan \f{spacing}.
	      Jika ada perubahan kode yang signifikan, Anda bisa mengusulkan ke penyusun \f{template}.
	      Keterangan lebih lanjut terkait cara kontribusi dapat dilihat di berkas \code{README.md} dan \code{CONTRIBUTING}.
\end{enumerate}

% %-----------------------------------------------------------------------------%
% \section{Melakukan \f{Cross-Reference} ke Suatu Bagian dalam Laporan}
% \label{sec:crossReference}
% %-----------------------------------------------------------------------------%
% Dengan menggunakan \gls{latex}, Anda tidak perlu lagi melakukan referensi ke suatu bagian atau objek dalam laporan secara manual.
% Anda cukup melakukan referensi ke bagian/gambar/kode/persamaan yang Anda inginkan dengan menggunakan perintah \code{\bslash{}ref}.
% Anda tidak perlu lagi mengubah referensi secara manual setiap kali ada perubahan letak pada bagian tersebut, karena \gls{latex}~akan melakukannya secara otomatis.
% Selain itu, pada berkas \acrfull{pdf} yang dihasilkan oleh \gls{latex}, referensi tersebut akan memiliki \f{link} yang langsung mengarahkan pembaca ke posisi objek atau bagian yang direferensikan.
% Untuk melakukan \f{cross-reference}, pertama kali tandai bagian yang ingin Anda referensikan dengan menggunakan suatu label, melalui perintah \code{\bslash{}label\{...:.....\}}.
% Label tidak boleh mengandung spasi. Berikut ini adalah konvensi penamaan label dan cara melakukan referensi yang digunakan dalam \f{template} ini:
% \begin{itemize}
%       \item \code{\bslash{}label\{bab:[nomorBab]\}} untuk sebuah bab. \\
%             Contoh: \code{\bslash{}label\{bab:3\}} \\
%             Cara referensi: \code{\bslash{}bab\~\bslash{}ref\{bab:3\}} \\
%             Hasil referensi: \bab~\ref{bab:3}.
%       \item \code{\bslash{}label\{sec:[....]\}} untuk sebuah subbab. \\
%             Contoh: \code{\bslash{}label\{sec:crossReference\}} \\
%             Cara referensi: \code{\bslash{}sect\~\bslash{}ref\{sec:crossReference\}} \\
%             Hasil referensi: \sect~\ref{sec:crossReference}.
%       \item \code{\bslash{}label\{appendix:[....]\}} untuk sebuah bab/subbab lampiran. \\
%             Contoh: \code{\bslash{}label\{appendix:changelog\}} \\
%             Cara referensi: \code{\bslash{}apdx\~\bslash{}ref\{appendix:changelog\}} \\	Hasil referensi: \apdx~\ref{appendix:changelog}.
%       \item \code{\bslash{}label\{equ:[....]\}} untuk sebuah persamaan matematis. \\
%             Contoh: \code{\bslash{}label\{equ:matriks\}} \\
%             Cara referensi: \code{\bslash{}equ\~\bslash{}ref\{equ:matriks\}} \\
%             Hasil referensi: \equ~\ref{equ:matriks}.
%       \item \code{\bslash{}label\{fig:[....]\}} untuk sebuah gambar. \\
%             Contoh: \code{\bslash{}label\{fig:testGambar\}} \\
%             Cara referensi: \code{\bslash{}pic\~\bslash{}ref\{fig:testGambar\}} \\
%             Hasil referensi: \pic~\ref{fig:testGambar}.
%       \item \code{\bslash{}label\{tab:[....]\}} untuk sebuah tabel. \\
%             Contoh: \code{\bslash{}label\{tab:\tab1\}} \\
%             Cara referensi: \code{\bslash{}tab\~\bslash{}ref\{tab:tab1\}} \\
%             Hasil referensi: \tab~\ref{tab:long}.
%       \item Untuk sebuah kode sumber, label diletakkan sebagai argumen dari \code{\bslash{}lstinputlisting} seperti: \code{\bslash{}lstinputlisting[..., label=code:...]}. \\
%             Contoh: \code{\bslash{}lstinputlisting[language=Java, caption=Kode sampel Java, label=code:java]} \\
%             Cara referensi: \code{\bslash{}lst\~\bslash{}ref\{code:java\}} \\
%             Hasil referensi: \lst~\ref{code:java}.
% \end{itemize}

% %-----------------------------------------------------------------------------%
% \section{Menambahkan Kode Program}
% \label{sec:codeListing}
% % Hal baru di template 2017
% %-----------------------------------------------------------------------------%
% % v2.2.1: tutorial dipindah dari Bab 3 ke Bab 2
% Pada \gls{latex}, kode program seringkali disebut \f{listing}.
% \f{Syntax highlighting} kini sudah bisa dilakukan secara otomatis oleh \f{library} yang ada di \gls{latex}.
% Sudah tidak perlu lagi membuat skrip manual untuk menambahkan \f{syntax highlighting} sendiri.
% \lst~\ref{code:java} adalah contoh kode program (\f{listing}) Java yang dicetak oleh \gls{latex}.

% \lstinputlisting[language=Java, caption=Kode sampel Java yang cukup panjang, label=code:java]{assets/codes/2-sample.java}

% Sintaks untuk memasukkan kode program ke dalam dokumen \gls{latex}~adalah sebagai berikut:

% \begin{lstlisting}[language={[latex]tex}, caption=Meng]
% \lstinputlisting[language=Java, caption=Kode sampel Java yang cukup panjang, label=code:java]{assets/codes/2-sample.java}
% \end{lstlisting}

% Terdapat tiga argumen yang digunakan pada perintah \code{\bslash{}lstinputlisting}:
% \begin{itemize}
%       \item \code{language} digunakan untuk menentukan bahasa pemrograman yang digunakan.
%             Untuk menggunakan suatu dialek bahasa pemrograman yang berbeda dari \f{default},
%             misalkan versi Python3 dari Python,
%             gunakan perintah \code{language=\{[3]Python\}}.
%       \item \code{caption} digunakan untuk memberikan \f{caption} pada kode program.
%             Argumen ini sifatnya opsional, jika ada, maka \f{caption} akan ditampilkan di bawah kode program.
%             Jika argumen ini tidak ada, maka \f{caption} tidak akan ditampilkan dan kode tidak bisa masuk ke daftar kode program.
%       \item \code{label} digunakan untuk memberikan label pada kode program untuk rujukan di dalam dokumen (\f{cross-reference}).
%             Argumen ini tidak boleh didefinisikan jika argumen \code{caption} tidak didefinisikan.
% \end{itemize}

% Terdapat empat kelompok bahasa pemrograman (dan dialek) yang didukung oleh implementasi \code{listings} pada \f{template} ini, yaitu:

% \begin{itemize}
%       \item \bo{Bahasa pemrograman yang didukung secara \f{default} oleh \code{listings}} (menurut \cite{latex:source_code_listings}): \\
%             \code{ABAP}, \code{ACSL}, \code{Ada}, \code{Algol}, \code{Ant}, \code{Awk}, \code{bash}, \code{Basic}, \code{C++}, \code{C}, \code{Caml},
%             \code{Clean}, \code{Cobol}, \code{Comal}, \code{command.com} (Windows Batch), \code{csh}, \code{Delphi}, \code{Eiffel}, \code{Elan},
%             \code{erlang}, \code{Euphoria}, \code{Fortran}, \code{GCL}, \code{Go} (golang), \code{Gnuplot}, \code{Haskell}, \code{HTML}, \code{IDL},
%             \code{inform}, \code{Java}, \code{JVMIS}, \code{ksh}, \code{Lisp}, \code{Logo}, \code{Lua}, \code{make}, \code{Mathematica}, \code{Matlab},
%             \code{Mercury}, \code{MetaPost}, \code{Miranda}, \code{Mizar}, \code{ML}, \code{Modelica}, \code{Modula-2}, \code{MuPAD}, \code{NASTRAN},
%             \code{Oberon-2}, \code{OCL}, \code{Octave}, \code{Oz}, \code{Pascal}, \code{Perl}, \code{PHP}, \code{PL/I}, \code{Plasm}, \code{POV},
%             \code{Prolog}, \code{Promela}, \code{PSTricks}, \code{Python}, \code{R}, \code{Reduce}, \code{Rexx}, \code{RSL}, \code{Ruby}, \code{S},
%             \code{SAS}, \code{Scilab}, \code{sh}, \code{SHELXL}, \code{Simula}, \code{SQL}, \code{tcl}, \code{TeX}, \code{VBScript}, \code{Verilog},
%             \code{VHDL}, \code{VRML}, \code{XML}, dan \code{XSLT}.
%       \item \bo{Dialek yang didukung secara \f{default} oleh \code{listings}} (menurut \cite{latex:source_code_listings}, diambil beberapa contoh):
%             \begin{itemize}
%                   \item Dialek Assembly: \code{[Motorola68k]\{Assembler\}}, \code{[x86masm]\{Assembler\}},
%                   \item Dialek Awk: \code{[gnu]\{Awk\}} (GNU Awk), \code{[POSIX]\{Awk\}},
%                   \item Dialek C: \code{[ANSI]\{C\}} (default), \code{[Handel]\{C\}}, \code{[Objective]\{C\}} (Objective-C), \code{[Sharp]\{C\}} (C\#),
%                   \item Dialek Caml: \code{[Objective]\{Caml\}} (OCaml), \code{[light]\{Caml\}} (default),
%                   \item Dialek C++: \code{[11]\{C++\}}, \code{[ANSI]\{C++\}}, \code{[GNU]\{C++\}}, \code{[Visual]\{C++\}} (Visual C++),
%                         \code{[ISO]\{C++\}} (default),
%                   \item Dialek Java: \code{[]\{Java\}} (default), \code{[AspectJ]\{Java\}},
%                   \item Dialek Pascal: \code{[Borland6]\{Pascal\}}, \code{[XSC]\{Pascal\}}, \code{[Standard]\{Pascal\}} (default),
%                   \item Dialek Python: \code{[2]\{Python\}} (default, Python 2), \code{[3]\{Python\}} (Python 3),
%                   \item Dialek TeX: \code{[LaTeX]\{TeX\}} (LaTeX), \code{[AlLaTeX]\{TeX\}}, \code{[plain]\{TeX\}} (plain TeX, default),
%                   \item Dialek tcl: \code{[]\{Tcl\}} (default Tcl), \code{[tk]\{Tcl\}} (Tcl/Tk).
%             \end{itemize}
%       \item \bo{Bahasa pemrograman yang ditambahkan pada \f{template} ini}: \\
%             \code{ABS}, \code{Acceleo}, \code{Batch}, \code{Clojure}, \code{CSS}, \code{D}, \code{Dart}, \code{Docker}, \code{F\#} (FSharp),
%             \code{GDScript} (Godot), \code{GLSL}, \code{Groovy}, \code{HLSL}, \code{JavaScript}, \code{Julia}, \code{Kotlin}, \code{Markdown},
%             \code{PowerShell}, \code{Rust}, \code{Scala}, \code{Scheme}, \code{Solidity}, \code{Swift}, \code{TOML}, \code{TypeScript}, dan \code{YAML}.
%       \item \bo{Dialek yang ditambahkan pada \f{template} ini}:
%             \begin{itemize}
%                   \item Dialek HTML: \code{[v5]\{HTML\}} (HTML5),
%                   \item Dialek Java: \code{[v9]\{Java\}} (Java 9 Modules), \code{[ContextJ]\{Java\}}, \code{[DeltaJ]\{Java\}}, dan \code{[FOP]\{Java\}}.
%             \end{itemize}
% \end{itemize}

% Berikut adalah contoh penggunaan dialek bahasa pemrograman dalam menuliskan kode program.
% Dalam kasus ini, \lst~\ref{code:python2} merupakan kode Python versi 2.
% Sedangkan, \lst~\ref{code:python3} merupakan kode Python versi 3.
% Perbedaan antara kedua versi tersebut cukup signifikan, salah satunya adalah pada perintah \code{print}.
% Pada Python 2, perintah \code{print} tidak memerlukan tanda kurung karena merupakan sebuah \f{keyword},
% sedangkan pada Python 3, perintah \code{print} memerlukan tanda kurung karena merupakan sebuah \f{function}.

% \lstinputlisting[language=Python, caption=Kode Python 2, label=code:python2]{assets/codes/2-sample-python2.py}

% % \lstinputlisting[language={[3]Python}, caption=Kode Python 3, label=code:python3]{assets/codes/2-sample-python3.py}

% Kode yang mencetak \lst~\ref{code:python2} dan \lst~\ref{code:python3} adalah sebagai berikut:

% \begin{lstlisting}[language={[latex]tex}]
% \lstinputlisting[language=Python, caption=Kode Python 2, label=code:python2]{assets/codes/2-sample-python2.py}
% \lstinputlisting[language={[3]Python}, caption=Kode Python 3, label=code:python3]{assets/codes/2-sample-python3.py}
% \end{lstlisting}

% Secara \f{default}, dialek yang digunakan untuk Python pada \f{library} \code{listings} adalah Python 2.
% Sehingga untuk mencetak kode Python 3, perlu digunakan dialek Python 3 (\code{\{[3]Python\}}).

% Kode program yang dicetak oleh \gls{latex} bersifat \f{auto-wrapped}.
% Jika suatu baris kode program melebihi batas lebar halaman,
% maka kode program tersebut akan dipindahkan ke baris berikutnya.
% \f{Auto-wrapped} ini berguna agar Anda tidak perlu memberikan \f{line break} manual pada kode Anda,
% Anda bisa menyampaikan kode program Anda apa adanya.

% \bo{Catatan}: Jangan lupa untuk menjelaskan kode melalui paragraf,
% terutama pada bagian-bagian yang perlu penjelasan lebih.
% Setiap kode perlu dijelaskan, kalau bisa rujuk ke setiap baris, karena belum tentu pembaca mau membaca kode Anda.
% Tetapi, pembaca tetap perlu mengetahui ide di balik kode yang Anda buat, dan mengapa kode tersebut dibuat.


% %-----------------------------------------------------------------------------%
% \section{\code{thesis.tex}}
% \label{sec:thesis-tex}
% %-----------------------------------------------------------------------------%
% Berkas \code{thesis.tex} berisi seluruh berkas \gls{latex} yang dibaca, jadi bisa dikatakan sebagai berkas utama.
% Dari berkas ini kita dapat mengatur bab apa saja yang ingin kita tampilkan dalam dokumen.


% %-----------------------------------------------------------------------------%
% \section{Direktori \code{config}}
% \label{sec:config-dir}
% %-----------------------------------------------------------------------------%
% Direktori \code{config} berisi berkas-berkas yang menyimpan konfigurasi variabel dan istilah-istilah yang bisa dimodifikasi sesuai dengan kebutuhan tugas akhir.

% %-----------------------------------------------------------------------------%
% \subsection{\code{settings.tex}}
% \label{sec:settings-tex}
% %-----------------------------------------------------------------------------%
% Berkas \code{settings.tex} berguna untuk mempermudah pembuatan beberapa template standar.
% Anda diminta untuk menuliskan judul laporan, nama, NPM, dan hal-hal lain yang dibutuhkan untuk pembuatan template.

% %-----------------------------------------------------------------------------%
% \subsection{\code{istilah.tex}}
% \label{sec:istilah-tex}
% %-----------------------------------------------------------------------------%
% Berkas \code{istilah.tex} digunakan untuk mencatat istilah-istilah yang digunakan.
% Fungsinya hanya untuk memudahkan penulisan.
% Pada beberapa kasus, ada kata-kata yang harus selalu muncul dengan tercetak miring atau tercetak tebal.
% Anda juga bisa menggunakan berkas ini untuk mencatat istilah atau akronim khusus yang perlu dimunculkan di Daftar Istilah.
% Penggunaan lebih lanjut terkait berkas \code{\_internals/istilah.tex} untuk menyimpan istilah atau akronim ada di \sect~\ref{sec:glossary}.
% Dengan menjadikan kata-kata tersebut sebagai sebuah perintah \gls{latex}~tentu akan mempercepat dan mempermudah pengerjaan laporan.

% %-----------------------------------------------------------------------------%
% \subsection{\code{references.bib}}
% \label{sec:references-bib}
% %-----------------------------------------------------------------------------%
% Berkas \code{references.bib} berisi seluruh daftar referensi yang digunakan dalam
% laporan.
% Anda bisa membuat model daftar referensi lain dengan menggunakan BibTeX.
% Untuk menambahkan referensi dengan format BibTeX, Anda bisa mengisi berkas \code{references.bib}.
% Untuk merujuk pada salah satu referensi yang ada, gunakan perintah \code{\bslash{}cite},
% e.g. \code{\bslash{}cite\{book:sample\}} yang akan akan memunculkan \cite{book:sample}.
% Informasi lebih lanjut mengenai referensi bisa dilihat di \sect~\ref{sec:references-bib}.
% Untuk mempelajari bibtex lebih lanjut, silahkan buka \url{http://www.bibtex.org/Format}.


% %-----------------------------------------------------------------------------%
% \section{Direktori \code{\_internals}}
% \label{sec:internals}
% %-----------------------------------------------------------------------------%
% Direktori \code{\_internals} berisi halaman-halaman dan \f{styling} yang tidak perlu diubah untuk penggunaan normal dari template ini.
% \f{Styling} bisa diubah jika diperlukan untuk menyesuaikan beberapa fitur template dengan kebutuhan tugas akhir, atau untuk menyesuaikan dengan aturan terbaru yang dirilis oleh Universitas Indonesia.

% %-----------------------------------------------------------------------------%
% \subsection{\code{hype.indonesia.tex}}
% \label{sec:hype-indonesia-tex}
% %-----------------------------------------------------------------------------%
% Berkas \code{hype.indonesia.tex} berisi cara pemenggalan beberapa kata dalam bahasa Indonesia.
% \gls{latex}~memiliki algoritma untuk memenggal kata-kata sendiri, namun untuk beberapa kasus algoritma ini memenggal dengan cara yang salah.
% Untuk memperbaiki pemenggalan yang salah inilah cara pemenggalan yang benar ditulis dalam berkas \f{hype.indonesia.tex}.

% %-----------------------------------------------------------------------------%
% \subsection{\code{uithesis.sty}}
% \label{sec:uithesis.sty}
% %-----------------------------------------------------------------------------%
% Berkas \code{uithesis.sty} berisi konfigurasi inti dari \f{layoutting} untuk \f{template} ini.
% Secara umum, Anda tidak perlu mengubah apapun pada berkas ini.
% Akan tetapi, untuk kasus-kasus lanjutan, seperti menambahkan daftar konten \f{custom} atau menyalakan dukungan terhadap \f{multi-language}, Anda bisa mengubahnya secara langsung pada \code{uithesis.sty}.
% Salah satu contohnya adalah ketika Anda ingin mendefinisikan daftar suatu jenis objek baru, seperti yang dicontohkan pada \sect~\ref{sec:addCustomContentList}.
% Atau bisia juga ketika Anda ingin mengganti tipe referensi, seperti yang dicontohkan pada \sect~\ref{sec:bibtexChangeFormat}.
% Jika Anda memiliki feedback maupun ingin berkontribusi terhadap perbaikan \f{layout}, selama ke arah yang sesuai dengan ketentuan Peraturan Rektor UI terkait format Tugas Akhir, Anda bisa mengubah berkas ini dan berkas lainnya yang terkait lalu membuat Merge Request di repositori.
% Keterangan lebih lanjut terkait cara kontribusi dapat dilihat di berkas \code{README.md} dan \code{CONTRIBUTING}.


% %-----------------------------------------------------------------------------%
% \section{Direktori \code{src/00-frontMatter}}
% \label{sec:frontMatter-backMatter-tex}
% %-----------------------------------------------------------------------------%
% Direktori \code{src/00-frontMatter} berisi bagian depan yang memuat halaman-halaman administratif untuk laporan ilmiah Anda.
% Sedangkan direktori \code{src/99-backMatter} berisikan berkas-berkas lampiran.
% Berikut adalah daftar berkas yang tersedia di \code{src/00-frontMatter}:
% \begin{enumerate}
% 	\item \code{pernyataanOrisinalitas.tex} untuk halaman pernyataan orisinalitas.
% 	      Berlaku untuk semua tipe dokumen kecuali Laporan Kerja Praktik dan Kampus Merdeka.
% 	\item \code{pengesahanKP.tex} untuk halaman pengesahan spesifik tipe dokumen Laporan Kerja Praktik.
% 	\item \code{pengesahanMBKM.tex} untuk halaman pengesahan spesifik tipe dokumen Kampus Merdeka.
% 	\item \code{pengesahanSidang.tex} untuk halaman pengesahan sidang.
% 	      Berlaku untuk semua tipe dokumen kecuali laporan ilmiah mahasiswa S3 (Disertasi), Laporan Kerja Praktik, dan Kampus Merdeka.
% 	\item \code{pengesahanSidangS3.tex} untuk halaman pengesahan sidang khusus mahasiswa S3.
% 	\item \code{kataPengantar.tex} untuk kata pengantar.
% 	      Berlaku untuk semua tipe dokumen kecuali Laporan Kerja Praktik dan Kampus Merdeka.
% 	\item \code{persetujuanPublikasi.tex} untuk halaman persetujuan publikasi karya intelektual.
% 	      Berlaku untuk semua tipe dokumen kecuali Laporan Kerja Praktik dan Kampus Merdeka.
% 	\item \code{abstrak.tex} untuk halaman abstrak berbahasa Indonesia.
% 	\item \code{abstract.tex} untuk halaman abstrak berbahasa Inggris.
% \end{enumerate}
% Umumnya, Anda hanya perlu mengisi bagian-bagian seperti Abstrak dan Kata Pengantar.
% Berkas sisanya berisi kode yang akan menghasilkan halaman-halaman terkait secara otomatis, sehingga hanya bisa diubah jika diperlukan penyesuaian, misal ukuran \f{line spacing}.


% %-----------------------------------------------------------------------------%
% \section{Direktori \code{src/01-body}}
% \label{sec:bab-tex}
% %-----------------------------------------------------------------------------%
% Direktori ini berisi isi laporan yang Anda tulis.
% Setiap nama berkas e.g. bab1.tex merepresentasikan bab dimana tulisan tersebut akan muncul.
% Sebagai contoh, kode dimana tulisan ini dibaut berada dalam berkas dengan nama \code{bab4.tex}.
% Ada enam buah berkas yang telah disiapkan untuk mengakomodir enam bab dari laporan Anda, diluar bab kesimpulan dan saran.
% Jika Anda tidak membutuhkan sebanyak itu, silahkan hapus kode dalam berkas \code{thesis.tex} yang memasukan berkas \gls{latex}~yang tidak dibutuhkan;
% contohnya perintah \code{\bslash{}include\{bab6.tex\}} merupakan kode untuk memasukan berkas \code{bab6.tex} kedalam laporan.



%-----------------------------------------------------------------------------%
% Akhir Bab 1
%-----------------------------------------------------------------------------%
