%-----------------------------------------------------------------------------%
\chapter{\babLima}
\label{bab:5}
%-----------------------------------------------------------------------------%

%-----------------------------------------------------------------------------%
\section{Eksperimen tanpa tahap / modul \textit{Search and Resolve}}
\label{sec:fol_and_cnf_ablation}
%-----------------------------------------------------------------------------%

\begin{figure}
	\centering
	\includegraphics[width=\textwidth]{assets/pics/Ablation_FOL_CNF.drawio.png}
	\captionof{figure}{Kerangka kerja Aristotle tanpa tahapan \textit{Search and Resolve}}
	\label{fig:fol_and_cnf_figure}
\end{figure}

Eksperimen ini dilakukan untuk melihat perbandingan performa \textit{framework Aristotle} dengan tahap akhir dan tanpa tahap akhir \textit{Search and Resolve}, khususnya pada model Qwen yang diduga mengalami penurunan akurasi pada tahap akhir. Eksperimen ini dilakukan dengan menggantikan modul \textit{Search and Resolve} dengan \textit{Chain-of-Thought} (CoT) \textit{prompting} untuk mengeveluasi \textit{output} dari tahap sebelumnya

\begin{table}[h]
	\centering
	\caption{Hasil Eksperimen tanpa Tahap Akhir \textit{Search \& Resolve} dan Resolusi dengan CoT \textit{Prompting}}
	% >{\centering\arraybackslash} centers the text inside the X column.
	\begin{tabularx}{\textwidth}{l >{\centering\arraybackslash}X >{\centering\arraybackslash}X >{\centering\arraybackslash}X}
		\toprule
		                       & \textbf{Qwen2.5 7B-Instruct-GGUF} & \textbf{SEA-LION v3-Llama-8B-GGUF} & \textbf{SahabatAI v1-Llama-8B-GGUF} \\
		\midrule
		\quad FOL and CNF only & 56.80\%                           & 62.60\%                            & 56.20\%                             \\
		\bottomrule
	\end{tabularx}
\end{table}

%-----------------------------------------------------------------------------%
\section{Eksperimen tanpa tahap / modul \textit{Search and Resolve} dan \textit{Decomposition to CNF}}
\label{sec:fol_ablation}
%-----------------------------------------------------------------------------%

\begin{figure}
	\centering
	\includegraphics[width=\textwidth]{assets/pics/Ablation_FOL.drawio.png}
	\captionof{figure}{Kerangka kerja Aristotle hanya dengan tahapan \textit{Translation to FOL}}
	\label{fig:fol_figure}
\end{figure}

Eksperimen ini dilakukan untuk melihat perbandingan performa \textit{framework Aristotle} dengan tahap akhir dan hanya dengan tahap \textit{Translation to FOL}. Eksperimen ini dilakukan dengan menggunakan modul \textit{Translation to FOL} lalu dievaluasi \textit{output} dari tahap tersebut dengan \textit{Chain-of-Thought} (CoT) \textit{prompting}

\begin{table}[h]
	\centering
	\caption{Hasil Eksperimen hanya dengan Tahap Pertama \textit{Translation to FOL} dan Resolusi dengan CoT \textit{Prompting}}
	% >{\centering\arraybackslash} centers the text inside the X column.
	\begin{tabularx}{\textwidth}{l >{\centering\arraybackslash}X >{\centering\arraybackslash}X >{\centering\arraybackslash}X}
		\toprule
		               & \textbf{Qwen2.5 7B-Instruct-GGUF} & \textbf{SEA-LION v3-Llama-8B-GGUF} & \textbf{SahabatAI v1-Llama-8B-GGUF} \\
		\midrule
		\quad FOL only & 77.40\%                           & 84.60\%                            & 58.60\%                             \\
		\bottomrule
	\end{tabularx}
\end{table}

\section{Eksperimen dengan tahap \textit{Translation to FOL} dan dilanjutkan dengan Prolog}
\label{sec:prolog_ablation}

\begin{figure}
	\centering
	\includegraphics[width=\textwidth]{assets/pics/Ablation_prolog.drawio.png}
	\captionof{figure}{Kerangka kerja Aristotle hanya dengan tahapan \textit{Translation to FOL} dan dilanjutkan dengan bantuan prolog}
	\label{fig:prolog_figure}
\end{figure}

Eksperimen ini dilakukan untuk melihat perbandingan performa \textit{framework Aristotle} dengan tahap akhir dan hanya dengan tahap \textit{Translation to FOL} yang hasil \textit{output} dari tahap ini di-\textit{resolve} dengan bantuan Prolog. Prolog sendiri pun tidak dapat melakukan resolusi terhadap premis-premis natural dan premis-premis tersebut beragam bentuk kalimatnya, sehingga hanya mengandalkan regex parsing saja tidak cukup.

\begin{table}[h]
	\centering
	\caption{Hasil Eksperimen hanya dengan Tahap Pertama \textit{Translation to FOL} dan Resolusi dengan Prolog}
	% >{\centering\arraybackslash} centers the text inside the X column.
	\begin{tabularx}{\textwidth}{l >{\centering\arraybackslash}X >{\centering\arraybackslash}X >{\centering\arraybackslash}X}
		\toprule
		                      & \textbf{Qwen2.5 7B-Instruct-GGUF} & \textbf{SEA-LION v3-Llama-8B-GGUF} & \textbf{SahabatAI v1-Llama-8B-GGUF} \\
		\midrule
		\quad FOL with Prolog & 82.20\%                           & 93.80\%                            & 73.20\%                             \\
		\bottomrule
	\end{tabularx}
\end{table}

Dari ketiga hasil ablasi eksperimen tersebut, dapat disimpulkan bahwa dengan mentranslasikan premis-premis yang ada dari bahasa natural ke bahasa simbolik dalam bentuk FOL dan dilanjutkan dengan \textit{Prolog resolver} mempunyai akurasi yang lebih tinggi untuk dataset ProntoQA ketimbang menggunakan SLM untuk melakukan resolusi, baik pada tahapan \textit{Translation to FOL}, maupun pada tahapan \textit{Decomposition into CNF}, akan tetapi hal ini akan sulit dilakukan pada dataset yang lebih kompleks dari ProntoQA, seperti dataset ProofWriter (\cite{tafjord-etal-2021-proofwriter}) dan dataset LogicNLI (\cite{tian-etal-2021-diagnosing}) karena dataset ProntoQA hanya membutuh jawaban \textit{True} atau \textit{False} saja dan premis-premis tersebut tidak memerlukan tahapan yang kompleks, seperti skolemisasi