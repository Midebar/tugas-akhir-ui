%-----------------------------------------------------------------------------%
\chapter{Landasan Teori}
\label{bab:2}
%-----------------------------------------------------------------------------%

Bab ini membangun kerangka teoretis yang mendasari analisis kemampuan \textit{Large Language Models} (LLM) dalam melakukan inferensi logika, khususnya dalam konteks Bahasa Indonesia. Pembahasan mencakup prinsip formal dari (\textit{First-Order Logic}), transformasi bentuk normal (PNF, Skolemisasi, CNF), serta arsitektur \textit{Neuro-Symbolic} "Aristotle".

%-----------------------------------------------------------------------------%
\section{Aturan Inferensi dan Resolusi}
\label{sec:inferenceRules}
%-----------------------------------------------------------------------------%

\subsection{Aturan Inferensi Klasik}
Sistem ini secara implisit mencakup aturan-aturan logika klasik:
\begin{itemize}
	\item \textbf{Modus Ponens:} Jika $P \rightarrow Q$ dan $P$, maka $Q$.
	\item \textbf{Modus Tollens:} Jika $P \rightarrow Q$ dan $\neg Q$, maka $\neg P$.
	\item \textbf{Silogisme Hipotetis:} Jika $P \rightarrow Q$ dan $Q \rightarrow R$, maka $P \rightarrow R$.
	\item \textbf{Silogisme Disjungtif:} Jika $P \lor Q$ dan $\neg P$, maka $Q$.
\end{itemize}

\subsection{Prinsip Resolusi dan Unifikasi}
Dalam pembuktian teorema aturan inferensi, aturan-aturan di atas dapat digeneralisasi menjadi satu aturan tunggal yang disebut \textbf{Resolusi}.

Untuk resolusi memerlukan proses \textbf{Unifikasi}. Unifikasi adalah proses mencari substitusi $\theta$ (pemetaan variabel ke term) sehingga dua literal menjadi identik.
\begin{equation}
	\frac{L_1 \lor A, \quad \neg L_2 \lor B}{(A \lor B)\theta}
\end{equation}
Dimana $\theta$ adalah \textit{Most General Unifier} (MGU) dari $L_1$ dan $L_2$.

Contoh implementasi:
\begin{itemize}
	\item Premis: "Semua orang menyukai Budi" ($\forall x Loves$($x, Budi$)).
	\item Query: "Apakah Ali menyukai Budi?" ($\neg Loves$($Ali, Budi$)).
	\item Unifikasi: Substitusi $\{x$ / $Ali\}$ membuat $Loves$($x, Budi$) berlawanan langsung dengan $\neg Loves$($Ali, Budi$).
	\item Hasil: Klausa kosong ($\perp$), yang berarti terbukti terjadi kontradiksi, sehingga Ali menyukai Budi.
\end{itemize}

%-----------------------------------------------------------------------------%
\section{(\textit{First-Order Logic})}
\label{sec:FOL}
%-----------------------------------------------------------------------------%

FOL atau sering disebut Kalkulus Predikat, berfungsi sebagai fondasi untuk merepresentasikan pengetahuan secara tidak ambigu. Berbeda dengan Logika Proposisi yang memperlakukan pernyataan sebagai objek yang tak terbagi, FOL memodelkan dunia sebagai domain objek, properti, dan hubungan antar objek tersebut.

\subsection{Definisi dan Sintaks}
Secara formal, FOL adalah sistem logika deduktif yang memperluas logika proposisi dengan memperkenalkan:
\begin{itemize}
	\item \textbf{Variabel ($x, y, z$):} Simbol yang mewakili objek dalam suatu domain.
	\item \textbf{Predikat ($P$($x,y$)):} Fungsi yang memetakan objek ke nilai kebenaran (\textit{True/False}).
	\item \textbf{Kuantor ($\forall, \exists$):} Operator yang menentukan \textit{range} variabel terhadap domain.
\end{itemize}

Alfabet dari FOL ($\Sigma$) terdiri dari:
\begin{enumerate}
	\item \textbf{Simbol Logika:}
	      \begin{itemize}
		      \item Konektif: $\neg$ (Negasi), $\land$ (Konjungsi), $\lor$ (Disjungsi), $\rightarrow$ (Implikasi).
		      \item Kuantor Universal ($\forall$): Dibaca "Untuk semua".
		      \item Kuantor Eksistensial ($\exists$): Dibaca "Terdapat".
	      \end{itemize}
	\item \textbf{Simbol Non-Logika:} Konstanta (objek spesifik seperti "Wumpus", "Merah") dan Predikat
\end{enumerate}

\subsection{Semantik dan Interpretasi}
Validitas dalam FOL ditentukan oleh sebuah \textbf{Model} $\mathcal{M} = \langle \mathcal{D}, \mathcal{I} \rangle$, di mana:
\begin{itemize}
	\item $\mathcal{D}$ adalah \textbf{Domain} (misalnya, himpunan semua entitas dalam basis data kependudukan).
	\item $\mathcal{I}$ adalah \textbf{Interpretasi} yang memetakan simbol konstanta ke elemen di $\mathcal{D}$ dan simbol predikat ke relasi di $\mathcal{D}$.
\end{itemize}

Dalam konteks Bahasa Indonesia, struktur kalimat \textit{Subjek-Predikat-Objek} dapat dipetakan ke dalam FOL, namun tantangan muncul pada ambiguitas konteks. Misalnya, kalimat "Gajah itu besar" bisa berarti konstanta ($Big(\text{gajah}_1)$) atau aturan universal ($\forall x (Elephant(x) \rightarrow Big(x))$), juga misalnya subjek plural dan singular yang perlu di normalisasi. Modul dekomposisi dalam penelitian ini bertugas menyelesaikan ambiguitas ini menjadi representasi FOL yang benar.

%-----------------------------------------------------------------------------%
\section{Bentuk Normal: Standardisasi untuk Algoritma}
\label{sec:normalForms}
%-----------------------------------------------------------------------------%

Algoritma penalaran, khususnya resolusi, tidak dapat bekerja secara efisien pada formula FOL yang tidak terstruktur atau tidak \textit{strict}. Formula harus dikonversi ke dalam bentuk standar atau \textit{Normal Forms}.

\subsection{Bentuk Normal Prenex (PNF)}
Langkah pertama standardisasi adalah \textit{Prenex Normal Form} (PNF). Sebuah formula dikatakan dalam bentuk PNF jika seluruh kuantor ditarik ke depan (prefix), meninggalkan bagian bebas-kuantor (matriks) di belakangnya.

Bentuk umum:
\begin{equation}
	Q_1 x_1 Q_2 x_2 \dots Q_n x_n \; \mathcal{M}
\end{equation}
Dimana $Q_i \in \{ \forall, \exists \}$ dan $\mathcal{M}$ adalah matriks tanpa kuantor. Contoh transformasi dalam Bahasa Indonesia:
\begin{itemize}
	\item \textit{Asli:} "Tidak benar bahwa semua orang menyukai durian." ($\neg \forall x Likes(x, Durian)$).
	\item \textit{PNF:} "Terdapat seseorang yang tidak menyukai durian." ($\exists x \neg Likes(x, Durian)$).
\end{itemize}

\subsection{Skolemisasi (Skolemization)}
Skolemisasi adalah proses menghilangkan \textbf{Kuantor Eksistensial} ($\exists$) untuk menghasilkan formula yang \textit{equisatisfiable} (memiliki keterpenuhan yang setara). Kuantor eksistensial digantikan oleh \textbf{Konstanta Skolem} atau \textbf{Fungsi Skolem}.

\textbf{Algoritma:}
\begin{enumerate}
	\item Jika $\exists y$ tidak berada dalam lingkup $\forall$, ganti $y$ dengan konstanta baru $c$.
	\item Jika $\exists y$ berada dalam lingkup $\forall x_1, \dots, \forall x_k$, ganti $y$ dengan fungsi $f(x_1, \dots, x_k)$.
\end{enumerate}

Contoh kasus:
\begin{quote}
	"Setiap warga negara memiliki sebuah NIK." \\
	FOL: $\forall x$ ($Citizen$($x$) $\rightarrow \exists y$ ($NIK$($y$) $\land Has$($x, y$))) \\
	Skolemisasi: $\forall x$ ($Citizen$($x$) $\rightarrow$ ($NIK$(nikOf($x$)) $\land Has$($x$, nikOf($x$))))
\end{quote}
Di sini, nikOf($x$) adalah fungsi Skolem yang memetakan setiap warga ke NIK spesifik mereka.

\subsection{Bentuk Normal Konjungtif (CNF)}
\textit{Conjunctive Normal Form} (CNF) adalah representasi akhir yang dibutuhkan oleh mesin inferensi. CNF adalah konjungsi dari klausa-klausa, di mana setiap klausa adalah disjungsi dari literal.
\begin{equation}
	(L_{1,1} \lor \dots \lor L_{1,n}) \land (L_{2,1} \lor \dots \lor L_{2,m})
\end{equation}

%-----------------------------------------------------------------------------%
\section{Paradigma Neuro-Symbolic}
\label{sec:neuroSymbolicIntro}
%-----------------------------------------------------------------------------%

Integrasi LLM dengan penalaran logika formal, sering disebut juga sebagai \textit{Neuro-Symbolic AI}, merupakan pergeseran penting dalam riset \textit{artificial intelligence}. Komponen "neuro" (model \textit{deep learning} seperti Qwen atau Llama) unggul dalam pengenalan pola, namun memiliki keterbatasan dalam melakukan inferensi multi-langkah yang \textit{strict} dan deterministik (\cite{saparov2023languagemodelsgreedyreasoners}). Keterbatasan ini menjadi poin penting pada bahasa dengan sumber daya rendah hingga menengah seperti Bahasa Indonesia, di mana korpus pelatihan untuk penalaran kompleks jauh lebih sedikit dibandingkan Bahasa Inggris.

Penelitian ini mengadopsi kerangka kerja yang tidak hanya mengandalkan prediksi token probabilitas, melainkan memisahkan proses menjadi translasi simbolik, pencarian terstruktur, dan resolusi formal, sebagaimana diusulkan dalam arsitektur \textbf{Aristotle} (\cite{Aristotle25}).
%-----------------------------------------------------------------------------%
\section{Kerangka Kerja Aristotle}
\label{sec:aristotleFramework}
%-----------------------------------------------------------------------------%

Penelitian ini mengimplementasikan \textit{framework} \textbf{Aristotle} (\cite{Aristotle25}) yang membagi proses inferensi menjadi empat modul utama:

\subsection{Logika Translasi (\textit{Logical Translation})}
Modul ini menggunakan LLM \textit{open-weight} untuk menerjemahkan premis bahasa natural menjadi FOL
\begin{itemize}
	\item \textbf{Input:} Data poin dalam bahasa Indonesia
	\item \textbf{Proses:} Menerjemahkan data poin menjadi 3 bagian, sesuai dengan formula, yaitu premis dari konteks $\rightarrow$ fakta dan aturan, pertanyaan dari konteks $\rightarrow$ konjektur
\end{itemize}

\subsection{Dekomposer (\textit{Decomposer})}
Modul ini menggunakan LLM \textit{open-weight} untuk dekomposisi aturan yang sudha berbentuk FOL ke dalam PNF, CNF, dan Skolemisasi
\begin{itemize}
	\item \textbf{Input:} data poin yang sudah melewati Logical Translation dalam bentuk FOL
	\item \textbf{Proses:} Dekomposisi data poin yang sudha menjadi FOL ke dalam bentuk NF, lalu Skolemisasi, dan terakhir CNF.
\end{itemize}

\subsection{Penyelesai Pencarian (\textit{Search Router})}
Mesin inti inferensi terdiri dari dua sub-komponen:
\begin{enumerate}
	\item \textbf{\textit{Logical Search Router}:} Memilih klausa mana yang relevan untuk diproses. Hal ini mencegah terjadinya ledakan kombinatorial yang mengakibatkan proses komputasi yang lama dan pencarian bukti yang sesuai dengan pembuatan set untuk menyimpan premis yang sudah ditelusuri dan memberi batasan banyaknya ronde dalam pencarian bukti
	\item \textbf{\textit{Logical Resolver}:} Menerapkan aturan resolusi secara iteratif hingga ditemukan kontradiksi (untuk pembuktian salah) atau hingga ruang pencarian habis.
\end{enumerate}

\subsection{Resolusi (\textit{Resolver})}
Sistem memverifikasi sebuah hipotesis $S$ melalui dua jalur paralel dengan menerapkan pembuktian dengan kontradiksi (\textit{Proof By Contradiction})
\begin{itemize}
	\item \textbf{Jalur 1 :} Asumsikan $S$. Jika ditemukan kontradiksi, maka $S$ adalah \textbf{Benar}.
	\item \textbf{Jalur 2 :} Asumsikan $\neg S$ (Negasi S). Jika ditemukan kontradiksi, maka $S$ adalah \textbf{Benar}.
\end{itemize}