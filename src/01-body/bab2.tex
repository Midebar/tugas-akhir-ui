%-----------------------------------------------------------------------------%
\chapter{Landasan Teori}
\label{bab:2}
%-----------------------------------------------------------------------------%

Bab ini membangun kerangka teoretis yang mendasari analisis kemampuan \textit{Large Language Models} (LLM) dalam melakukan penalaran logika. Pembahasan mencakup prinsip formal dari \textit{First-Order Logic} (FOL), prosedur konversi ke \textit{Conjunctive Normal Form} (CNF), serta algoritma Resolusi yang menjadi mesin utama dalam \textit{framework} Aristotle.

%-----------------------------------------------------------------------------%
\section{\textit{First-Order Logic} (FOL)}
\label{sec:fol}
%-----------------------------------------------------------------------------%
First-Order Logic (FOL), atau Kalkulus Predikat, adalah sistem formal yang memperluas logika proposisi dengan memperkenalkan variabel, fungsi, dan kuantor untuk merepresentasikan objek dan relasi di dunia nyata.

\subsection{Sintaks: Alfabet dan Aturan Pembentukan}
Mengacu pada definisi standar dalam buku Discrete Mathematics and Its Applications karya Kenneth H. Rosen \cite{rosen-discrete}, sintaks FOL dibangun dari komponen-komponen berikut:
\begin{itemize}
	\item Simbol Logis:
	      \begin{itemize}
		      \item Konektif: $\neg$ (Negasi), $\land$ (Konjungsi), $\lor$ (Disjungsi), $\rightarrow$ (Implikasi), $\leftrightarrow$ (Bikondisional).
		      \item Kuantor: $\forall$ (Universal), $\exists$ (Eksistensial).
		      \item Variabel: $x, y, z, \dots$
	      \end{itemize}
	\item Simbol Non-Logis:
	      \begin{itemize}
		      \item Konstanta: Simbol yang merepresentasikan objek spesifik (misalnya, "Alice", "42").
		      \item Fungsi ($f$($x_1$, $\dots$, $x_n$)): Memetakan objek ke objek lain. Contoh: $AyahDari$($Budi$).
		      \item Predikat ($P$($x_1$, $\dots$, $x_n$)): Fungsi yang memetakan tuple objek ke nilai kebenaran (True/False). Contoh: $Suka$($Budi, Apel$).
	      \end{itemize}
	\item Grammar (Tata Bahasa):
	      \begin{itemize}
		      \item Term: Sebuah variabel, konstanta, atau fungsi yang diterapkan pada term lain.
		      \item Formula Atomik: Predikat yang diterapkan pada term. Ini adalah unit terkecil yang memiliki nilai kebenaran.
		      \item \textit{Well-Formed Formulas} (WFFs): Formula yang disusun secara rekursif dari formula atomik menggunakan konektif dan kuantor. Rosen menekankan pentingnya cakupan kuantor (scope), di mana variabel dalam cakupan kuantor disebut bound variable (terikat), sedangkan yang di luar adalah free variable (bebas).
	      \end{itemize}
\end{itemize}

\subsection{Semantik: Interpretasi dan Kebenaran}
Kebenaran sebuah kalimat FOL ditentukan oleh sebuah Interpretasi ($\mathcal{I}$) atas Domain ($\mathcal{D}$) yang tidak kosong. Interpretasi memetakan:
\begin{itemize}
	\item Konstanta ke elemen di $\mathcal{D}$.
	\item Memetakan predikat $n$-ary ke relasi $n$-ary di $\mathcal{D}$.
	\item Sebuah formula $\alpha$ dikatakan Benar di bawah interpretasi $\mathcal{I}$ (ditulis $\mathcal{I} \models \alpha$) jika fakta di dunia nyata sesuai dengan struktur kalimat tersebut.
\end{itemize}

%-----------------------------------------------------------------------------%
\section{(Konsekuensi Logis \textit{(Entailment)})}
\label{sec:FOL}
%-----------------------------------------------------------------------------%
Tujuan utama sistem berbasis pengetahuan adalah menarik kesimpulan baru dari premis yang ada. Konsep ini diformalkan sebagai Entailment ($KB \models \alpha$). Menurut \cite{huth-ryan-lics2}, $KB \models \alpha$ berlaku jika dan hanya jika untuk setiap interpretasi di mana $KB$ bernilai benar, $\alpha$ juga pasti bernilai benar.Dalam konteks komputasi, memeriksa semua interpretasi adalah mustahil. Oleh karena itu, kita menggunakan Inference Rules (aturan inferensi) sintaksis untuk membuktikan validitas tanpa memeriksa semantik satu per satu.

%-----------------------------------------------------------------------------%
\section{Resolusi dan Unifikasi}
\label{sec:normalForms}
%-----------------------------------------------------------------------------%
Metode inferensi utama yang digunakan dalam modul "Resolver" Aristotle adalah Resolusi. Teknik ini diperkenalkan oleh J.A. Robinson pada tahun 1965 dalam makalah seminalnya "A Machine-Oriented Logic Based on the Resolution Principle".\cite{DBLP:journals/jacm/Robinson65}

\subsection{\textit{Proof by Refutation} (Pembuktian Kontradiksi)}
Robinson membuktikan bahwa untuk menguji $KB \models \alpha$, kita cukup membuktikan bahwa himpunan $KB \cup \{\neg \alpha\}$ adalah Unsatisfiable (tidak mungkin benar bersamaan). Jika kita dapat menurunkan klausa kosong ($\square$ atau False) dari himpunan tersebut, maka $\alpha$ terbukti benar.\cite{DBLP:journals/jacm/Robinson65}

\subsection{Konversi ke \textit{Conjunctive Normal Form} (CNF)}
Agar aturan resolusi dapat diterapkan, formula logika harus diubah ke bentuk standar yang disebut \textit{Conjunctive Normal Form} (CNF). Menurut \cite{fitting-fol-atp} merinci langkah-langkah algoritma konversi ini sebagai berikut:

\begin{enumerate}
	\item Eliminasi Implikasi: Ubah $A \rightarrow B$ menjadi $\neg A \lor B$.
	\item Geser Negasi ke Dalam: Gunakan hukum De Morgan dan aturan $\neg \forall x P \equiv \exists x \neg P$.
	\item Standardisasi Variabel: Ubah nama variabel agar unik untuk setiap kuantor (misal: $\forall x P(x) \lor \forall x Q(x)$ menjadi $\forall x P(x) \lor \forall y Q(y)$).
	\item Prenex Normal Form: Pindahkan semua kuantor universal ($\forall$) ke depan formula.
	\item Skolemisasi: Menghilangkan kuantor eksistensial ($\exists$). Variabel $y$ yang terikat oleh $\exists$ diganti dengan Fungsi Skolem $f(x)$ yang bergantung pada variabel universal sebelumnya.Contoh: $\forall x \exists y (Parent(x, y))$ menjadi $\forall x (Parent(x, f(x)))$.
	\item Distribusi \& CNF: Gunakan aturan distributif untuk mendapatkan bentuk konjungsi dari klausa (AND of ORs).
\end{enumerate}

\subsection{Aturan Resolusi Robinson dan Unifikasi}
Prinsip Resolusi Robinson menyederhanakan berbagai aturan inferensi klasik (seperti Modus Ponens dan Modus Tollens) menjadi satu aturan tunggal:
$$\frac{C_1 \lor L_1, \quad C_2 \lor L_2}{(C_1 \lor C_2)\theta}$$
Di mana: $L_1$ dan $L_2$ adalah literal yang dapat dibuat saling berlawanan (komplemen). $\theta$ adalah substitusi variabel yang dihasilkan oleh algoritma Unifikasi.

Unifikasi adalah proses mencari Most General Unifier (MGU) $\theta$ yang membuat dua atom sintaksis menjadi identik.
Contoh: Unifikasi $P(x)$ dan $\neg P(Alex)$ menghasilkan $\theta = \{x/Alex\}$. Tanpa unifikasi, resolusi pada FOL tidak mungkin dilakukan karena variabel pada premis yang berbeda harus diselaraskan.