%-----------------------------------------------------------------------------%
\chapter{\babSatu}
\label{bab:1}
%-----------------------------------------------------------------------------%

Bab ini menjelaskan latar belakang, rumusan masalah, tujuan penelitian, batasan, manfaat, tinjauan singkat metodologi, serta sistematika penulisan dari penelitian ini.


%-----------------------------------------------------------------------------%
\section{Latar Belakang}
\label{sec:latBel}
%-----------------------------------------------------------------------------%
Perkembangan Model Bahasa Besar (Large Language Models, LLM) dalam beberapa tahun terakhir telah menunjukkan kemampuan mengesankan pada berbagai tugas pemrosesan bahasa alami seperti penerjemahan, ringkasan, dan tanya-jawab. 
Salah satu aspek penting yang menjadi fokus kajian adalah kemampuan \emph{penalaran logis}, kemampuan menarik inferensi yang benar dari kumpulan premis dan aturan. 
Sebagian besar benchmark penalaran dan reasoning (misalnya ProofWriter, LogicNLI, ProntoQA) ditujukan pada bahasa Inggris. Namun demikian, studi terhadap kemampuan penalaran LLM pada bahasa lain, termasuk Bahasa Indonesia, masih terbatas. 
Perbedaan struktur linguistik, tanda baca, serta isu translasi otomatis dapat memengaruhi performa model ketika berpindah domain bahasa.
Penelitian ini bertujuan untuk mengevaluasi kemampuan penalaran logis LLM pada dataset berbahasa Indonesia dan menguji efektivitas pipeline reasoning berbasis modul: translasi/normalisasi, dekomposisi aturan, pencarian klausa komplement, serta resolusi logika. 
Fokus utama adalah mendapatkan gambaran secara kuantitatif maupun kualitatif tentang batas kemampuan LLM pada skenario berbahasa Indonesia dan menyajikan rekomendasi untuk eksperimen selanjutnya.

%-----------------------------------------------------------------------------%
\section{Rumusan Masalah}
\label{sec:rumusan}
%-----------------------------------------------------------------------------%
Berdasarkan latar belakang tersebut, rumusan masalah penelitian ini adalah sebagai berikut:
\begin{enumerate}
  \item Sejauh mana LLM berperforma dalam tugas penalaran logis pada dataset berbahasa Indonesia?
  \item Bagaimana perbandingan performa antara model open-source berparameter rendah (low-resource) dengan model yang lebih besar dalam konteks penalaran logis Bahasa Indonesia?
  \item Seberapa efektif implementasi pipeline \textit{logical reasoning framework} dalam meningkatkan kinerja penalaran pada dataset berbahasa Indonesia?
  \item Apa tantangan utama (mis. tokenisasi, konteks budaya, format data) yang muncul ketika memindahkan dataset/pujian penalaran dari bahasa Inggris ke bahasa Indonesia?
\end{enumerate}

%-----------------------------------------------------------------------------%
\section{Tujuan Penelitian}
\label{sec:tujuan}
%-----------------------------------------------------------------------------%
Tujuan penelitian ini dibagi menjadi tujuan umum dan tujuan khusus:
\begin{itemize}
  \item \textbf{Tujuan umum:} Mengevaluasi kemampuan penalaran logis LLM pada dataset berbahasa Indonesia.
  \item \textbf{Tujuan khusus:}
        \begin{enumerate}
          \item Mengukur performa beberapa model (lokal/kuantisasi vs. pembanding) pada tugas penalaran.
          \item Mengevaluasi kontribusi pipeline \emph{translate $\rightarrow$ decompose $\rightarrow$ search $\rightarrow$ resolve}.
          \item Mengidentifikasi sumber kesalahan (error analysis) dan memberikan rekomendasi.
        \end{enumerate}
\end{itemize}

%-----------------------------------------------------------------------------%
\section{Batasan Penelitian}
\label{sec:batasan}
%-----------------------------------------------------------------------------%
Agar ruang lingkup penelitian terukur, maka penelitian ini memiliki batasan:
\begin{itemize}
  \item Dataset: fokus pada dataset berformat proof-style / NLI yang tersedia untuk Bahasa Indonesia (adaptasi/terjemahan dari ProofWriter/LogicNLI serta korpus NLI lokal).
  \item Model: fokus pada model open-source yang dapat dijalankan secara lokal (termasuk versi kuantisasi), dan bila perlu perbandingan terhadap model akses API.
  \item Evaluasi: metrik utama adalah akurasi final, dan analisis kualitatif terhadap kasus salah.
  \item Lingkup eksperimen dibatasi pada dev/test split dataset mulai dari 10\% hingga 100\% (tergantung sumber daya komputasi) dari datatset itu sendiri.
\end{itemize}

%-----------------------------------------------------------------------------%
\section{Manfaat Penelitian}
\label{sec:manfaat}
%-----------------------------------------------------------------------------%
Hasil penelitian diharapkan memberikan:
\begin{itemize}
  \item Kontribusi ilmiah mengenai kemampuan penalaran LLM pada Bahasa Indonesia.
  \item Panduan teknis untuk penelitian lanjutan dan praktisi (pemilihan model, pipeline).
  \item Dataset dan skrip eksperimen yang dapat direplikasi.
\end{itemize}

%-----------------------------------------------------------------------------%
\section{Metodologi Penelitian (Ringkas)}
\label{sec:metodologi}
%-----------------------------------------------------------------------------%
Metodologi inti penelitian disusun sebagai pipeline modular yang terdiri dari langkah-langkah berikut:
\begin{enumerate}
  \item \textbf{Persiapan data:} Seleksi dan normalisasi dataset berbahasa Indonesia; jika perlu, terjemahan dan adaptasi dari dataset proof-style (format: premis, aturan, conjecture).
  \item \textbf{Backend LLM:} Konfigurasi backend model (lokal / kuantisasi / API) dan tokenizers.
  \item \textbf{Pipeline reasoning:} Implementasi modul:
        \begin{itemize}
          \item \texttt{translate\_decompose.py} — normalisasi + dekomposisi aturan,
          \item \texttt{negate.py} — pembalikan label / penanganan negasi,
          \item \texttt{search\_resolve.py} — pencarian klausa komplemen dan logic resolution,
          \item \texttt{evaluate.py} — perhitungan akurasi dan laporan.
        \end{itemize}
  \item \textbf{Eksperimen:} Jalankan pengujian, ukur akurasi, performa (waktu, memori) per jumlah split dataset, dan analisis kesalahan.
  \item \textbf{Analisis:} Klasifikasi kesalahan, bandingkan per-model, serta tarik rekomendasi.
\end{enumerate}

%-----------------------------------------------------------------------------%
\section{Sistematika Penulisan}
\label{sec:sistematika}
%-----------------------------------------------------------------------------%
Susunan laporan ini adalah sebagai berikut:
\begin{itemize}
  \item \textbf{Bab 1} -- Pendahuluan (latar belakang, rumusan masalah, tujuan, metodologi singkat).
  \item \textbf{Bab 2} -- Tinjauan Pustaka (teori terkait LLM, penalaran logis, dataset).
  \item \textbf{Bab 3} -- Metodologi Eksperimen (desain eksperimen dan implementasi pipeline).
  \item \textbf{Bab 4} -- Hasil dan Analisis.
  \item \textbf{Bab 5} -- Kesimpulan dan Saran.
\end{itemize}

%-----------------------------------------------------------------------------%
\section{Panduan Singkat: Sitasi dan Cross-Reference}
\label{sec:guide_citation}
%-----------------------------------------------------------------------------%
Untuk sitasi gunakan BibTeX seperti pada template; contoh pemanggilan:
\begin{itemize}
  \item Sitasi di dalam kalimat: \verb|Menurut \cite{author:year} ...|
  \item Sitasi di akhir kalimat: \verb|... sesuai kajian sebelumnya \citep{author:year}.|
\end{itemize}
Contoh cross-reference:
\begin{itemize}
  \item Referensi ke Bab: \verb|\label{bab:1}| kemudian \verb|\ref{bab:1}| (contoh: Bab~\ref{bab:1}).
  \item Referensi ke Sub-bab: \verb|\label{sec:metodologi}| lalu \verb|Section~\ref{sec:metodologi}|.
\end{itemize}

% Jika Anda ingin menandai bagian yang belum lengkap:
\todo{Lengkapi bagian dataset dan konfigurasi backend pada Bab 3.}

%-----------------------------------------------------------------------------%
% Akhir Bab 1
%-----------------------------------------------------------------------------%
