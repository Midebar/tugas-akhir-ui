%-----------------------------------------------------------------------------%
\chapter{Pendahuluan}
\label{bab:1}
%-----------------------------------------------------------------------------%

Bab ini memaparkan latar belakang, permasalahan, tujuan, batasan, manfaat, ringkasan metodologi, serta sistematika penulisan penelitian ini. Penelitian ini berfokus pada evaluasi kemampuan penalaran logis oleh \textit{Large Language Models} (LLM) yang bersifat \textit{open-weight} ketika bekerja pada dataset berbahasa Indonesia dengan \textit{framework translation-decomposition-searchresolve} juga perbandingannya dengan \textit{naive prompting}

\section{Latar Belakang}
\label{sec:latBel}

Perkembangan \textit{Large Language Model} (LLM) telah mendorong kemajuan signifikan pada berbagai tugas pemrosesan bahasa natural seperti penerjemahan, ringkasan, dan tanya-jawab. Namun, kemampuan LLM untuk melakukan penalaran logis, yaitu melakukan inferensi yang benar dari himpunan premis dan aturan formal, masih menghadapi kendala pada akurasi dari hasil inferensi, terutama di kasus yang memerlukan normalisasi, dekomposisi, pencarian bukti, dan resolusi logika.

Penelitian sebelumnya sudah dilakukan pada berbagai dataset menggunakan framework yang sama. Dataset yang digunakan sebagian besar dalam bahasa Inggris, sehingga studi terhadap kemampuan penalaran LLM pada bahasa lain, termasuk Bahasa Indonesia, masih terbatas. Perbedaan struktur linguistik, idiom, dan masalah tokenisasi serta kualitas terjemahan dapat memengaruhi performa model setelah adaptasi lintas bahasa.
Penelitian sebelumnya juga menggunakan \textit{proprietary} LLM yang bukan \textit{open-weight}, sehingga untuk mencari perbedaan dan perbandingan performa \textit{apple-to-apple} tidak memungkinkan.

Dari \textit{gap} penelitian yang sudah disebutkan, belum ada kajian untuk meneliti efektivitas \textit{framework} pada dataset berbahasa Indonesia. Penelitian ini menjadi penting untuk mengevaluasi kemampuan LLM dalam inferensi pada dataset berbahasa Indonesia dan mengeksplorasi sebuah \textit{framework} yang mengintegrasikan modul terjemahan dan normalisasi, dekomposisi logis, mekanisme pencarian bukti, serta resolusi logika. Diharapkan bahwa pendekatan terintegrasi ini dapat secara signifikan meningkatkan kemampuan penalaran LLM pada bahasa Indonesia.

\section{Rumusan Masalah}
\label{sec:rumusan}

Berdasarkan latar belakang di atas, rumusan masalah penelitian ini adalah:
\begin{enumerate}
      \item Bagaimana perbandingan performa antara model open-weight berparameter rendah dalam inferensi dataset berbahasa Indonesia?
      \item bagaimana efektivitas \textit{framework} \emph{translate $\rightarrow$ decompose $\rightarrow$ search $\rightarrow$ resolve} dalam meningkatkan akurasi inferensi pada data Bahasa Indonesia dibanding dengan penalaran secara langsung secara naive?
\end{enumerate}

\section{Tujuan Penelitian}
\label{sec:tujuan}

Berikut adalah tujuan dari penelitian sesuai dengan latar belakang dan rumusan masalah

\textbf{Tujuan:}
\begin{enumerate}
      \item Mengukur performa beberapa model pada tugas penalaran menggunakan metrik akurasi.
      \item Mengevaluasi kemampuan penalaran logika LLM pada dataset berbahasa Indonesia menggunakan \textit{framework} dan tanpa \textit{framework}, seperti \textit{naive prompting}
      \item Menyediakan dataset terjemahan, skrip eksperimen, dan laporan yang dapat digunakan penelitian selanjutnya.
\end{enumerate}

\section{Batasan Penelitian}
\label{sec:batasan}

Agar fokus dan ruang lingkup terukur, penelitian ini dibatasi sebagai berikut:
\begin{itemize}
      \item \textbf{Dataset:} Fokus pada dataset diterjemahkan ke Bahasa Indonesia, yaitu ProntoQA saja
      \item \textbf{Model:} Eksperimen menggunakan model open-weight yang dapat dijalankan lokal maupun server, khususnya dengan kuantisasi.
      \item \textbf{Evaluasi:} Metrik utama adalah akurasi jawaban akhir terhadap ground truth.
            % \item \textbf{Sumber daya:} Eksperimen disesuaikan dengan kapasitas komputasi dan biaya, sampling dimulai pada 10\% hingga 100\%.
\end{itemize}

Referensi dari framework dan metode, termasuk skrip seperti \texttt{translate\_decompose.py}, \texttt{negate.py}, dan \texttt{search\_resolve.py}, serta utilitas evaluasi tersedia pada repositori eksperimen yang menjadi inspirasi implementasi ini, yaitu pada repositori Aristotle.

\section{Manfaat Penelitian}
\label{sec:manfaat}

Penelitian ini diharapkan memberikan kontribusi yang bermakna bagi berbagai pihak:

\begin{itemize}
      \item \textbf{Bagi pengembangan ilmu pengetahuan:} Penelitian ini dapat menambah pemahaman tentang kemampuan penalaran logis LLM pada bahasa Indonesia, sebuah aspek yang masih jarang dikaji. Temuan ini dapat menjadi fondasi bagi penelitian lanjutan dalam evaluasi model bahasa pada tugas-tugas inferensi kompleks dalam bahasa lokal.
            
      \item \textbf{Bagi praktisi dan pengembang:} Hasil analisis perbandingan model dan efektivitas framework dapat menjadi panduan dalam memilih model open-weight yang tepat dan merancang pipeline pemrosesan bahasa untuk tugas inferensi logis berbahasa Indonesia, terutama dengan sumber daya komputasi terbatas.
            
      \item \textbf{Bagi komunitas penelitian terbuka:} Dataset ProntoQA yang diterjemahkan ke bahasa Indonesia, skrip eksperimen, serta laporan hasil penelitian akan dibagikan kepada publik. Kontribusi ini memungkinkan peneliti lain untuk mereplikasi, memvalidasi, dan melanjutkan penelitian dalam domain yang sama tanpa perlu melakukan terjemahan dan persiapan data dari awal.
\end{itemize}

%-----------------------------------------------------------------------------%
\section{Posisi Penelitian}
\label{sec:posisiPenelitian}
%-----------------------------------------------------------------------------%

\begin{figure}
      \centering
      \includegraphics[width=0.7\textwidth]{assets/pics/posisi_penelitian.png}
      \caption{Diagram posisi penelitian yang dilakukan}
      \label{fig:research_position}
\end{figure}

\todo{
      Jelaskan \pic~\ref{fig:research_position} di sini.
      Setiap gambar yang dimasukkan ke tugas akhir \bo{WAJIB} untuk dijelaskan oleh minimal satu paragraf.
}

%-----------------------------------------------------------------------------%
\section{Sistematika Penulisan}
\label{sec:sistematikaPenulisan}
%-----------------------------------------------------------------------------%
Sistematika penulisan laporan adalah sebagai berikut:
\begin{itemize}
      \item Bab 1 \babSatu \\
            Bab ini mencakup latar belakang, cakupan penelitian, dan pendefinisian masalah.
      \item Bab 2 \babDua \\
            Bab ini mencakup pemaparan terminologi dan teori yang terkait dengan penelitian berdasarkan hasil tinjauan pustaka yang telah digunakan, sekaligus memperlihatkan kaitan teori dengan penelitian.
      \item Bab 3 \babTiga \\
            Bab ini mencakup metodologi atau langkah-langkah yang digunakan untuk melakukan penelitian ini.
      \item Bab 4 \babEmpat \\
            Bab ini mencakup eksperimen dan hasil eksperimen penelitian serta analisis dari hasil eksperimen tersebut
      \item \kesimpulan \\
            Bab ini mencakup kesimpulan akhir penelitian dan saran untuk pengembangan berikutnya.
\end{itemize}
