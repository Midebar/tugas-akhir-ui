%-----------------------------------------------------------------------------%
\chapter{Pendahuluan}
\label{bab:1}
%-----------------------------------------------------------------------------%

Bab ini memaparkan latar belakang, permasalahan, tujuan, batasan, manfaat, ringkasan metodologi, serta sistematika penulisan penelitian ini. Penelitian berfokus pada evaluasi kemampuan penalaran logis oleh Model Bahasa Besar (Large Language Models, LLM) ketika bekerja pada dataset berbahasa Indonesia.

\section{Latar Belakang}
\label{sec:latBel}

Perkembangan \textit{Large Language Model} (LLM) telah mendorong kemajuan signifikan pada berbagai tugas pemrosesan bahasa natural seperti penerjemahan, ringkasan, dan tanya-jawab. Namun, kemampuan LLM untuk melakukan penalaran logis, yaitu menarik inferensi yang benar dari himpunan premis dan aturan formal, masih menghadapi kendala pada akurasi dan kebenaran dari hasil inferensi, terutama di kasus yang memerlukan normalisasi, dekomposisi, pencarian bukti, dan resolusi logika.

Sebagian besar dataset penalaran dibuat dalam bahasa Inggris, sehingga studi terhadap kemampuan penalaran LLM pada bahasa lain, termasuk Bahasa Indonesia, relatif terbatas. Perbedaan struktur linguistik, idiom, dan masalah tokenisasi serta kualitas terjemahan dapat memengaruhi performa model setelah adaptasi lintas bahasa. Oleh karena itu, diperlukan adaptasi dan evaluasi sistematis pada dataset berbahasa Indonesia serta investigasi textit{pipeline} yang menggabungkan modul terjemahan/normalisasi, dekomposisi aturan, mekanisme pencarian bukti, dan resolusi logika.

\section{Rumusan Masalah}
\label{sec:rumusan}

Berdasarkan latar belakang di atas, rumusan masalah penelitian ini adalah:
\begin{enumerate}
      \item Sejauh mana LLM mampu melakukan penalaran logis pada dataset berbahasa Indonesia?
      \item Bagaimana perbandingan performa antara model open-source berparameter rendah dalam inferensi dataset?
      \item Seberapa efektif pipeline \emph{translate $\rightarrow$ decompose $\rightarrow$ search $\rightarrow$ resolve} dalam meningkatkan akurasi inferensi pada data Bahasa Indonesia dibading dengan penalaran secara langsung secara naive?
      \item Apa saja sumber utama kegagalan (kesalahan translasi, tokenisasi, ambiguitas budaya/linguistik, format data) ketika memindahkan benchmark penalaran dari Bahasa Inggris ke Bahasa Indonesia?
\end{enumerate}

\section{Tujuan Penelitian}
\label{sec:tujuan}

\textbf{Tujuan umum:} Mengevaluasi dan memperbaiki kemampuan penalaran logis LLM pada dataset berbahasa Indonesia menggunakan pipeline.

\textbf{Tujuan khusus:}
\begin{enumerate}
      \item Mengukur performa beberapa model pada tugas penalaran menggunakan metrik akurasi dan analisis kesalahan.
      \item Mengidentifikasi dan mengkategorikan sumber kesalahan serta memberikan rekomendasi praktis untuk pengolahan data dan desain eksperimen berbahasa Indonesia.
      \item Menyediakan dataset terjemahan, skrip eksperimen, dan laporan replikasi yang dapat digunakan peneliti lain.
\end{enumerate}

\section{Batasan Penelitian}
\label{sec:batasan}

Agar fokus dan ruang lingkup terukur, penelitian ini dibatasi sebagai berikut:
\begin{itemize}
      \item \textbf{Dataset:} Fokus pada dataset \textit{proof-style} yang telah diterjemahkan ke Bahasa Indonesia, yaitu ProntoQA saja
      \item \textbf{Model:} Eksperimen menggunakan model open-source yang dapat dijalankan lokal maupun server, khususnya dengan kuantisasi.
      \item \textbf{Evaluasi:} Metrik utama adalah akurasi jawaban akhir terhadap ground truth.
      \item \textbf{Sumber daya:} Eksperimen disesuaikan dengan kapasitas komputasi, sampling dev/test split dimulai pada 10\% hingga 100\% tergantung ketersediaan.
\end{itemize}

Referensi dari pipeline dan metoode, termasuk skrip seperti \texttt{translate\_decompose.py}, \texttt{negate.py}, dan \texttt{search\_resolve.py}, serta utilitas evaluasi tersedia pada repositori eksperimen yang menjadi inspirasi implementasi ini, yaitu pada repositori Aristotle LaTeX.

\section{Manfaat Penelitian}
\label{sec:manfaat}

Hasil penelitian ini diharapkan memberi manfaat:
\begin{itemize}
      \item Bagi akademik: memperkaya literatur penalaran LLM pada bahasa Indonesia.
      \item Bagi praktisi: panduan pemilihan model dan desain pipeline untuk tugas inferensi logis berbahasa lokal.
      \item Bagi komunitas open-source: dataset dan skrip replikasi yang dapat mempercepat penelitian lanjutan.
\end{itemize}

%-----------------------------------------------------------------------------%
\section{Posisi Penelitian}
\label{sec:posisiPenelitian}
%-----------------------------------------------------------------------------%
\todo{
      Sebutkan posisi penelitian Anda. Ada baiknya jika Anda menggunakan gambar atau diagram.
      Template ini telah menyediakan contoh cara memasukkan gambar.
}

\begin{figure}
      \centering
      \includegraphics[width=0.4\textwidth]{assets/pics/makara.png}
      \caption{Penjelasan singkat terkait gambar.}
      \label{fig:research_position}
\end{figure}

\todo{
      Jelaskan \pic~\ref{fig:research_position} di sini.
      Setiap gambar yang dimasukkan ke tugas akhir \bo{WAJIB} untuk dijelaskan oleh minimal satu paragraf.
}

\section{Metodologi Singkat}
\label{sec:metodologi}

Pendekatan penelitian dirancang sebagai pipeline yang memadukan teknik pemrosesan teks dan mekanisme reasoning simbolik:
\begin{enumerate}
      \item \textbf{Persiapan Data:} Dataset yang ada di translasikan ke Bahasa Indonesia menggunakan model open-source
      \item \textbf{Dekomposisi Logis:} Mengurai aturan kompleks menjadi bentuk yang lebih sederhana ke dalam First Order Logic (FOL) dan menormalisasi ke Prenex Normal Form (PNF) atau Conjunctive Normal Form (CNF).
      \item \textbf{Inisialisasi Dua Jalur Pencarian:} Menegasikan konjektur atau pertanyaan untuk membentuk klausa komplemen yang akan dicari.
      \item \textbf{Search \& Resolve:} Melakukan pencarian klausa komplemen dan resolusi logika untuk menyimpulkan kebenaran atau identifikasi kontradiksi.
      \item \textbf{Evaluasi:} Mengevaluasi hasil akhir dengan menggabungkan jawaban dari dua jalur tersebut dan membandingkannya dengan ground truth dan di aggregasi menggunakan metrik akurasi.
\end{enumerate}

%-----------------------------------------------------------------------------%
\section{Sistematika Penulisan}
\label{sec:sistematikaPenulisan}
%-----------------------------------------------------------------------------%
Sistematika penulisan laporan adalah sebagai berikut:
\begin{itemize}
      \item Bab 1 \babSatu \\
            Bab ini mencakup latar belakang, cakupan penelitian, dan pendefinisian masalah.
      \item Bab 2 \babDua \\
            Bab ini mencakup pemaparan terminologi dan teori yang terkait dengan penelitian berdasarkan hasil tinjauan pustaka yang telah digunakan, sekaligus memperlihatkan kaitan teori dengan penelitian.
      \item Bab 3 \babTiga \\
            Bab ini mencakup metodologi atau langkah-langkah yang digunakan untuk melakukan penelitian ini.
      \item Bab 4 \babEmpat \\
            Bab ini mencakup eksperimen dan hasil eksperimen penelitian
      \item Bab 5 \babLima \\
            Bab ini mencakup analisis dari hasil eksperimen yang sudah dilakukan
      \item Bab 6 \kesimpulan \\
            Bab ini mencakup kesimpulan akhir penelitian dan saran untuk pengembangan berikutnya.
\end{itemize}

% %-----------------------------------------------------------------------------%
% \section{Melakukan \f{Cross-Reference} ke Suatu Bagian dalam Laporan}
% \label{sec:crossReference}
% %-----------------------------------------------------------------------------%
% Dengan menggunakan \gls{latex}, Anda tidak perlu lagi melakukan referensi ke suatu bagian atau objek dalam laporan secara manual.
% Anda cukup melakukan referensi ke bagian/gambar/kode/persamaan yang Anda inginkan dengan menggunakan perintah \code{\bslash{}ref}.
% Anda tidak perlu lagi mengubah referensi secara manual setiap kali ada perubahan letak pada bagian tersebut, karena \gls{latex}~akan melakukannya secara otomatis.
% Selain itu, pada berkas \acrfull{pdf} yang dihasilkan oleh \gls{latex}, referensi tersebut akan memiliki \f{link} yang langsung mengarahkan pembaca ke posisi objek atau bagian yang direferensikan.
% Untuk melakukan \f{cross-reference}, pertama kali tandai bagian yang ingin Anda referensikan dengan menggunakan suatu label, melalui perintah \code{\bslash{}label\{...:.....\}}.
% Label tidak boleh mengandung spasi. Berikut ini adalah konvensi penamaan label dan cara melakukan referensi yang digunakan dalam \f{template} ini:
% \begin{itemize}
%       \item \code{\bslash{}label\{bab:[nomorBab]\}} untuk sebuah bab. \\
%             Contoh: \code{\bslash{}label\{bab:3\}} \\
%             Cara referensi: \code{\bslash{}bab\~\bslash{}ref\{bab:3\}} \\
%             Hasil referensi: \bab~\ref{bab:3}.
%       \item \code{\bslash{}label\{sec:[....]\}} untuk sebuah subbab. \\
%             Contoh: \code{\bslash{}label\{sec:crossReference\}} \\
%             Cara referensi: \code{\bslash{}sect\~\bslash{}ref\{sec:crossReference\}} \\
%             Hasil referensi: \sect~\ref{sec:crossReference}.
%       \item \code{\bslash{}label\{appendix:[....]\}} untuk sebuah bab/subbab lampiran. \\
%             Contoh: \code{\bslash{}label\{appendix:changelog\}} \\
%             Cara referensi: \code{\bslash{}apdx\~\bslash{}ref\{appendix:changelog\}} \\	Hasil referensi: \apdx~\ref{appendix:changelog}.
%       \item \code{\bslash{}label\{equ:[....]\}} untuk sebuah persamaan matematis. \\
%             Contoh: \code{\bslash{}label\{equ:matriks\}} \\
%             Cara referensi: \code{\bslash{}equ\~\bslash{}ref\{equ:matriks\}} \\
%             Hasil referensi: \equ~\ref{equ:matriks}.
%       \item \code{\bslash{}label\{fig:[....]\}} untuk sebuah gambar. \\
%             Contoh: \code{\bslash{}label\{fig:testGambar\}} \\
%             Cara referensi: \code{\bslash{}pic\~\bslash{}ref\{fig:testGambar\}} \\
%             Hasil referensi: \pic~\ref{fig:testGambar}.
%       \item \code{\bslash{}label\{tab:[....]\}} untuk sebuah tabel. \\
%             Contoh: \code{\bslash{}label\{tab:\tab1\}} \\
%             Cara referensi: \code{\bslash{}tab\~\bslash{}ref\{tab:tab1\}} \\
%             Hasil referensi: \tab~\ref{tab:long}.
%       \item Untuk sebuah kode sumber, label diletakkan sebagai argumen dari \code{\bslash{}lstinputlisting} seperti: \code{\bslash{}lstinputlisting[..., label=code:...]}. \\
%             Contoh: \code{\bslash{}lstinputlisting[language=Java, caption=Kode sampel Java, label=code:java]} \\
%             Cara referensi: \code{\bslash{}lst\~\bslash{}ref\{code:java\}} \\
%             Hasil referensi: \lst~\ref{code:java}.
% \end{itemize}

% %-----------------------------------------------------------------------------%
% \section{Menambahkan Kode Program}
% \label{sec:codeListing}
% % Hal baru di template 2017
% %-----------------------------------------------------------------------------%
% % v2.2.1: tutorial dipindah dari Bab 3 ke Bab 2
% Pada \gls{latex}, kode program seringkali disebut \f{listing}.
% \f{Syntax highlighting} kini sudah bisa dilakukan secara otomatis oleh \f{library} yang ada di \gls{latex}.
% Sudah tidak perlu lagi membuat skrip manual untuk menambahkan \f{syntax highlighting} sendiri.
% \lst~\ref{code:java} adalah contoh kode program (\f{listing}) Java yang dicetak oleh \gls{latex}.

% \lstinputlisting[language=Java, caption=Kode sampel Java yang cukup panjang, label=code:java]{assets/codes/2-sample.java}

% Sintaks untuk memasukkan kode program ke dalam dokumen \gls{latex}~adalah sebagai berikut:

% \begin{lstlisting}[language={[latex]tex}, caption=Meng]
% \lstinputlisting[language=Java, caption=Kode sampel Java yang cukup panjang, label=code:java]{assets/codes/2-sample.java}
% \end{lstlisting}

% Terdapat tiga argumen yang digunakan pada perintah \code{\bslash{}lstinputlisting}:
% \begin{itemize}
%       \item \code{language} digunakan untuk menentukan bahasa pemrograman yang digunakan.
%             Untuk menggunakan suatu dialek bahasa pemrograman yang berbeda dari \f{default},
%             misalkan versi Python3 dari Python,
%             gunakan perintah \code{language=\{[3]Python\}}.
%       \item \code{caption} digunakan untuk memberikan \f{caption} pada kode program.
%             Argumen ini sifatnya opsional, jika ada, maka \f{caption} akan ditampilkan di bawah kode program.
%             Jika argumen ini tidak ada, maka \f{caption} tidak akan ditampilkan dan kode tidak bisa masuk ke daftar kode program.
%       \item \code{label} digunakan untuk memberikan label pada kode program untuk rujukan di dalam dokumen (\f{cross-reference}).
%             Argumen ini tidak boleh didefinisikan jika argumen \code{caption} tidak didefinisikan.
% \end{itemize}

% Terdapat empat kelompok bahasa pemrograman (dan dialek) yang didukung oleh implementasi \code{listings} pada \f{template} ini, yaitu:

% \begin{itemize}
%       \item \bo{Bahasa pemrograman yang didukung secara \f{default} oleh \code{listings}} (menurut \cite{latex:source_code_listings}): \\
%             \code{ABAP}, \code{ACSL}, \code{Ada}, \code{Algol}, \code{Ant}, \code{Awk}, \code{bash}, \code{Basic}, \code{C++}, \code{C}, \code{Caml},
%             \code{Clean}, \code{Cobol}, \code{Comal}, \code{command.com} (Windows Batch), \code{csh}, \code{Delphi}, \code{Eiffel}, \code{Elan},
%             \code{erlang}, \code{Euphoria}, \code{Fortran}, \code{GCL}, \code{Go} (golang), \code{Gnuplot}, \code{Haskell}, \code{HTML}, \code{IDL},
%             \code{inform}, \code{Java}, \code{JVMIS}, \code{ksh}, \code{Lisp}, \code{Logo}, \code{Lua}, \code{make}, \code{Mathematica}, \code{Matlab},
%             \code{Mercury}, \code{MetaPost}, \code{Miranda}, \code{Mizar}, \code{ML}, \code{Modelica}, \code{Modula-2}, \code{MuPAD}, \code{NASTRAN},
%             \code{Oberon-2}, \code{OCL}, \code{Octave}, \code{Oz}, \code{Pascal}, \code{Perl}, \code{PHP}, \code{PL/I}, \code{Plasm}, \code{POV},
%             \code{Prolog}, \code{Promela}, \code{PSTricks}, \code{Python}, \code{R}, \code{Reduce}, \code{Rexx}, \code{RSL}, \code{Ruby}, \code{S},
%             \code{SAS}, \code{Scilab}, \code{sh}, \code{SHELXL}, \code{Simula}, \code{SQL}, \code{tcl}, \code{TeX}, \code{VBScript}, \code{Verilog},
%             \code{VHDL}, \code{VRML}, \code{XML}, dan \code{XSLT}.
%       \item \bo{Dialek yang didukung secara \f{default} oleh \code{listings}} (menurut \cite{latex:source_code_listings}, diambil beberapa contoh):
%             \begin{itemize}
%                   \item Dialek Assembly: \code{[Motorola68k]\{Assembler\}}, \code{[x86masm]\{Assembler\}},
%                   \item Dialek Awk: \code{[gnu]\{Awk\}} (GNU Awk), \code{[POSIX]\{Awk\}},
%                   \item Dialek C: \code{[ANSI]\{C\}} (default), \code{[Handel]\{C\}}, \code{[Objective]\{C\}} (Objective-C), \code{[Sharp]\{C\}} (C\#),
%                   \item Dialek Caml: \code{[Objective]\{Caml\}} (OCaml), \code{[light]\{Caml\}} (default),
%                   \item Dialek C++: \code{[11]\{C++\}}, \code{[ANSI]\{C++\}}, \code{[GNU]\{C++\}}, \code{[Visual]\{C++\}} (Visual C++),
%                         \code{[ISO]\{C++\}} (default),
%                   \item Dialek Java: \code{[]\{Java\}} (default), \code{[AspectJ]\{Java\}},
%                   \item Dialek Pascal: \code{[Borland6]\{Pascal\}}, \code{[XSC]\{Pascal\}}, \code{[Standard]\{Pascal\}} (default),
%                   \item Dialek Python: \code{[2]\{Python\}} (default, Python 2), \code{[3]\{Python\}} (Python 3),
%                   \item Dialek TeX: \code{[LaTeX]\{TeX\}} (LaTeX), \code{[AlLaTeX]\{TeX\}}, \code{[plain]\{TeX\}} (plain TeX, default),
%                   \item Dialek tcl: \code{[]\{Tcl\}} (default Tcl), \code{[tk]\{Tcl\}} (Tcl/Tk).
%             \end{itemize}
%       \item \bo{Bahasa pemrograman yang ditambahkan pada \f{template} ini}: \\
%             \code{ABS}, \code{Acceleo}, \code{Batch}, \code{Clojure}, \code{CSS}, \code{D}, \code{Dart}, \code{Docker}, \code{F\#} (FSharp),
%             \code{GDScript} (Godot), \code{GLSL}, \code{Groovy}, \code{HLSL}, \code{JavaScript}, \code{Julia}, \code{Kotlin}, \code{Markdown},
%             \code{PowerShell}, \code{Rust}, \code{Scala}, \code{Scheme}, \code{Solidity}, \code{Swift}, \code{TOML}, \code{TypeScript}, dan \code{YAML}.
%       \item \bo{Dialek yang ditambahkan pada \f{template} ini}:
%             \begin{itemize}
%                   \item Dialek HTML: \code{[v5]\{HTML\}} (HTML5),
%                   \item Dialek Java: \code{[v9]\{Java\}} (Java 9 Modules), \code{[ContextJ]\{Java\}}, \code{[DeltaJ]\{Java\}}, dan \code{[FOP]\{Java\}}.
%             \end{itemize}
% \end{itemize}

% Berikut adalah contoh penggunaan dialek bahasa pemrograman dalam menuliskan kode program.
% Dalam kasus ini, \lst~\ref{code:python2} merupakan kode Python versi 2.
% Sedangkan, \lst~\ref{code:python3} merupakan kode Python versi 3.
% Perbedaan antara kedua versi tersebut cukup signifikan, salah satunya adalah pada perintah \code{print}.
% Pada Python 2, perintah \code{print} tidak memerlukan tanda kurung karena merupakan sebuah \f{keyword},
% sedangkan pada Python 3, perintah \code{print} memerlukan tanda kurung karena merupakan sebuah \f{function}.

% \lstinputlisting[language=Python, caption=Kode Python 2, label=code:python2]{assets/codes/2-sample-python2.py}

% % \lstinputlisting[language={[3]Python}, caption=Kode Python 3, label=code:python3]{assets/codes/2-sample-python3.py}

% Kode yang mencetak \lst~\ref{code:python2} dan \lst~\ref{code:python3} adalah sebagai berikut:

% \begin{lstlisting}[language={[latex]tex}]
% \lstinputlisting[language=Python, caption=Kode Python 2, label=code:python2]{assets/codes/2-sample-python2.py}
% \lstinputlisting[language={[3]Python}, caption=Kode Python 3, label=code:python3]{assets/codes/2-sample-python3.py}
% \end{lstlisting}

% Secara \f{default}, dialek yang digunakan untuk Python pada \f{library} \code{listings} adalah Python 2.
% Sehingga untuk mencetak kode Python 3, perlu digunakan dialek Python 3 (\code{\{[3]Python\}}).

% Kode program yang dicetak oleh \gls{latex} bersifat \f{auto-wrapped}.
% Jika suatu baris kode program melebihi batas lebar halaman,
% maka kode program tersebut akan dipindahkan ke baris berikutnya.
% \f{Auto-wrapped} ini berguna agar Anda tidak perlu memberikan \f{line break} manual pada kode Anda,
% Anda bisa menyampaikan kode program Anda apa adanya.

% \bo{Catatan}: Jangan lupa untuk menjelaskan kode melalui paragraf,
% terutama pada bagian-bagian yang perlu penjelasan lebih.
% Setiap kode perlu dijelaskan, kalau bisa rujuk ke setiap baris, karena belum tentu pembaca mau membaca kode Anda.
% Tetapi, pembaca tetap perlu mengetahui ide di balik kode yang Anda buat, dan mengapa kode tersebut dibuat.


%-----------------------------------------------------------------------------%
% Akhir Bab 1
%-----------------------------------------------------------------------------%
