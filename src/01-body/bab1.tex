%-----------------------------------------------------------------------------%
\chapter{Pendahuluan}
\label{bab:1}
%-----------------------------------------------------------------------------%

Bab ini memaparkan latar belakang, permasalahan, tujuan, batasan, manfaat, ringkasan metodologi, serta sistematika penulisan penelitian ini. Penelitian berfokus pada evaluasi kemampuan penalaran logis oleh Model Bahasa Besar (Large Language Models, LLM) ketika bekerja pada dataset berbahasa Indonesia.

\section{Latar Belakang}
\label{sec:latBel}

Perkembangan \textit{Large Language Model} (LLM) telah mendorong kemajuan signifikan pada berbagai tugas pemrosesan bahasa alami seperti penerjemahan, ringkasan, dan tanya-jawab. Namun, kemampuan LLM untuk melakukan penalaran logis, yaitu menarik inferensi yang benar dari himpunan premis dan aturan formal, masih menghadapi kendala baik pada akurasi maupun keandalan, terutama di kasus yang memerlukan normalisasi, dekomposisi, dan pencarian bukti yang kompleks, serta resolusi logika.

Sebagian besar dataset penalaran dibuat dalam bahasa Inggris, sehingga studi terhadap kemampuan penalaran LLM pada bahasa lain, termasuk Bahasa Indonesia, relatif terbatas. Perbedaan struktur linguistik, idiom, dan masalah tokenisasi serta kualitas terjemahan dapat memengaruhi performa model setelah adaptasi lintas bahasa. Oleh karena itu, diperlukan adaptasi dan evaluasi sistematis pada dataset berbahasa Indonesia serta investigasi pipeline yang menggabungkan modul terjemahan/normalisasi, dekomposisi aturan, mekanisme pencarian bukti, dan resolusi logika.

\section{Rumusan Masalah}
\label{sec:rumusan}

Berdasarkan latar belakang di atas, rumusan masalah penelitian ini adalah:
\begin{enumerate}
  \item Sejauh mana LLM mampu melakukan penalaran logis pada dataset berbahasa Indonesia?
  \item Bagaimana perbandingan performa antara model open-source berparameter rendah dalam inferensi dataset?
  \item Seberapa efektif pipeline \emph{translate $\rightarrow$ decompose $\rightarrow$ search $\rightarrow$ resolve} dalam meningkatkan akurasi inferensi pada data Bahasa Indonesia dibading dengan penalaran secara langsung secara naive?
  \item Apa saja sumber utama kegagalan (kesalahan translasi, tokenisasi, ambiguitas budaya/linguistik, format data) ketika memindahkan benchmark penalaran dari Bahasa Inggris ke Bahasa Indonesia?
\end{enumerate}

\section{Tujuan Penelitian}
\label{sec:tujuan}

\textbf{Tujuan umum:} Mengevaluasi dan memperbaiki kemampuan penalaran logis LLM pada dataset berbahasa Indonesia menggunakan pipeline.

\textbf{Tujuan khusus:}
\begin{enumerate}
  \item Mengukur performa beberapa model pada tugas penalaran menggunakan metrik akurasi dan analisis kesalahan.
  \item Mengidentifikasi dan mengkategorikan sumber kesalahan serta memberikan rekomendasi praktis untuk pengolahan data dan desain eksperimen berbahasa Indonesia.
  \item Menyediakan dataset terjemahan, skrip eksperimen, dan laporan replikasi yang dapat digunakan peneliti lain.
\end{enumerate}

\section{Batasan Penelitian}
\label{sec:batasan}

Agar fokus dan ruang lingkup terukur, penelitian ini dibatasi sebagai berikut:
\begin{itemize}
  \item \textbf{Dataset:} Fokus pada dataset \textit{proof-style} yang telah diterjemahkan ke Bahasa Indonesia, yaitu ProntoQA saja
  \item \textbf{Model:} Eksperimen menggunakan model open-source yang dapat dijalankan lokal maupun server, khususnya dengan kuantisasi.
  \item \textbf{Evaluasi:} Metrik utama adalah akurasi jawaban akhir terhadap ground truth.
  \item \textbf{Sumber daya:} Eksperimen disesuaikan dengan kapasitas komputasi, sampling dev/test split dimulai pada 10\% hingga 100\% tergantung ketersediaan.
\end{itemize}

Referensi dari pipeline dan metoode, termasuk skrip seperti \texttt{translate\_decompose.py}, \texttt{negate.py}, dan \texttt{search\_resolve.py}, serta utilitas evaluasi tersedia pada repositori eksperimen yang menjadi inspirasi implementasi ini, yaitu pada repositori Aristotle LaTeX.

\section{Manfaat Penelitian}
\label{sec:manfaat}

Hasil penelitian ini diharapkan memberi manfaat:
\begin{itemize}
  \item Bagi akademik: memperkaya literatur penalaran LLM pada bahasa Indonesia.
  \item Bagi praktisi: panduan pemilihan model dan desain pipeline untuk tugas inferensi logis berbahasa lokal.
  \item Bagi komunitas open-source: dataset dan skrip replikasi yang dapat mempercepat penelitian lanjutan.
\end{itemize}

%-----------------------------------------------------------------------------%
\section{Posisi Penelitian}
\label{sec:posisiPenelitian}
%-----------------------------------------------------------------------------%
\todo{
  Sebutkan posisi penelitian Anda. Ada baiknya jika Anda menggunakan gambar atau diagram.
  Template ini telah menyediakan contoh cara memasukkan gambar.
}

\begin{figure}
  \centering
  \includegraphics[width=0.4\textwidth]{assets/pics/makara.png}
  \caption{Penjelasan singkat terkait gambar.}
  \label{fig:research_position}
\end{figure}

\todo{
  Jelaskan \pic~\ref{fig:research_position} di sini.
  Setiap gambar yang dimasukkan ke tugas akhir \bo{WAJIB} untuk dijelaskan oleh minimal satu paragraf.
}

\section{Metodologi Singkat}
\label{sec:metodologi}

Pendekatan penelitian dirancang sebagai pipeline yang memadukan teknik pemrosesan teks dan mekanisme reasoning simbolik:
\begin{enumerate}
  \item \textbf{Persiapan Data:} Dataset yang ada di translasikan ke Bahasa Indonesia menggunakan model open-source
  \item \textbf{Dekomposisi Logis:} Mengurai aturan kompleks menjadi bentuk yang lebih sederhana ke dalam First Order Logic (FOL) dan menormalisasi ke Prenex Normal Form (PNF) atau Conjunctive Normal Form (CNF).
  \item \textbf{Inisialisasi Dua Jalur Pencarian:} Menegasikan konjektur atau pertanyaan untuk membentuk klausa komplemen yang akan dicari.
  \item \textbf{Search \& Resolve:} Melakukan pencarian klausa komplemen dan resolusi logika untuk menyimpulkan kebenaran atau identifikasi kontradiksi.
  \item \textbf{Evaluasi:} Mengevaluasi hasil akhir dengan menggabungkan jawaban dari dua jalur tersebut dan membandingkannya dengan ground truth dan di aggregasi menggunakan metrik akurasi.
\end{enumerate}

%-----------------------------------------------------------------------------%
\section{Sistematika Penulisan}
\label{sec:sistematikaPenulisan}
%-----------------------------------------------------------------------------%
Sistematika penulisan laporan adalah sebagai berikut:
\begin{itemize}
  \item Bab 1 \babSatu \\
        Bab ini mencakup latar belakang, cakupan penelitian, dan pendefinisian masalah.
  \item Bab 2 \babDua \\
        Bab ini mencakup pemaparan terminologi dan teori yang terkait dengan penelitian berdasarkan hasil tinjauan pustaka yang telah digunakan, sekaligus memperlihatkan kaitan teori dengan penelitian.
  \item Bab 3 \babTiga \\
        Apa itu Bab 3?
  \item Bab 4 \babEmpat \\
        Apa itu Bab 4?
  \item Bab 5 \babLima \\
        Apa itu Bab 5?
  \item Bab 6 \kesimpulan \\
        Bab ini mencakup kesimpulan akhir penelitian dan saran untuk pengembangan berikutnya.
\end{itemize}

%-----------------------------------------------------------------------------%
\section{Panduan Singkat: Sitasi dan Cross-Reference}
\label{sec:guide_citation}
%-----------------------------------------------------------------------------%
Untuk sitasi gunakan BibTeX seperti pada template; contoh pemanggilan:
\begin{itemize}
  \item Sitasi di dalam kalimat: \verb|Menurut \cite{author:year} ...|
  \item Sitasi di akhir kalimat: \verb|... sesuai kajian sebelumnya \citep{author:year}.|
\end{itemize}
Contoh cross-reference:
\begin{itemize}
  \item Referensi ke Bab: \verb|\label{bab:1}| kemudian \verb|\ref{bab:1}| (contoh: Bab~\ref{bab:1}).
  \item Referensi ke Sub-bab: \verb|\label{sec:metodologi}| lalu \verb|Section~\ref{sec:metodologi}|.
\end{itemize}

% Jika Anda ingin menandai bagian yang belum lengkap:
\todo{Lengkapi bagian dataset dan konfigurasi backend pada Bab 3.}

%-----------------------------------------------------------------------------%
% Akhir Bab 1
%-----------------------------------------------------------------------------%
