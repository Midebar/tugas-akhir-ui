%---------------------------------------------------------------
\chapter{\kesimpulan}
\label{bab:kesimpulan}
%---------------------------------------------------------------
Bab ini merangkum temuan-temuan utama yang diperoleh dari rangkaian eksperimen validasi kerangka kerja Neuro-Symbolic pada dataset logika berbahasa Indonesia. Kesimpulan disusun untuk menjawab rumusan masalah yang telah ditetapkan pada Bab 1, diikuti dengan saran strategis untuk pengembangan penelitian selanjutnya.

\section{Kesimpulan} \label{sec:kesimpulan}

Berdasarkan hasil evaluasi pada model \textit{open-weight} dengan parameter rendah (7B-9B) yang dikuantisasi pada dataset berbahasa Indonesia, penelitian ini menghasilkan tiga kesimpulan utama:

\begin{enumerate} 
	\item \textbf{Mekanisme Penalaran Logis pada Bahasa Indonesia:}
	      Implementasi \textit{framework translation-decomposition-search-resolve} pada data berbahasa Indonesia berhasil mengubah paradigma permasalahan dari "probabilitas linguistik" menjadi "validitas simbolik". Tantangan utama dalam mekanisme ini bukan terletak pada kemampuan mesin \textit{solver} untuk menyelesaikan logika, melainkan pada tahap \textbf{Translasi Logika} (\textit{Logical Translation}). Ditemukan bahwa struktur kalimat Bahasa Indonesia yang implisit (sering tanpa penanda subjek eksplisit) memerlukan \textit{prompt engineering} yang sangat spesifik untuk mencegah model menghasilkan halusinasi sintaks. Keberhasilan framework ini sangat bergantung pada kemampuan model untuk memisahkan struktur "Fakta" dan "Aturan" secara presisi sebelum masuk ke mesin inferensi.
	      
	\item \textbf{Perbandingan Performa Antar Model Open-Weight:} \\
	      Terdapat divergensi hasil yang signifikan antara model global dan model regional/nasional dalam tugas translasi simbolik ini:
	      \begin{itemize}
		      \item \textbf{SEA-LION v3 (Regional)} mencatat performa terbaik dengan akurasi \textbf{81.60\%}. Model ini menunjukkan keseimbangan terbaik antara pemahaman instruksi dalam Bahasa Indonesia dan kepatuhan terhadap format logika formal.
		      \item \textbf{Sahabat-AI v1 (Nasional)} menunjukkan performa yang stabil (\textbf{61.20\%}), mengungguli model global dalam memahami nuansa semantik lokal, meskipun masih di bawah SEA-LION dalam konsistensi sintaks kompleks.
		      \item \textbf{Qwen2.5 (Global)}, meskipun dikenal unggul dalam *coding*, mengalami kegagalan katastropik (\textbf{14.00\%}) pada metode ini. Kegagalan ini disebabkan oleh ketidakmampuan model untuk menjawab sesuai penalaran logis ketika dipaksa mengikuti format ketat, menandakan bahwa kapabilitas \textit{instruction following} tidak selalu berbanding lurus dengan kemampuan penalaran logis dalam konteks bahasa tertentu.
	      \end{itemize}
	\item \textbf{Efektivitas Aristotle vs Naive Prompting:} \\
	      Penerapan framework Aristotle terbukti \textbf{meningkatkan reliabilitas} jika dan hanya jika model memiliki kapabilitas \textit{instruction following} yang kuat dalam bahasa target.
	      \begin{itemize}
		      \item Pada model \textbf{SEA-LION}, penggunaan framework Aristotle meningkatkan akurasi dari 76.80\% (\textit{Naive}) menjadi \textbf{81.60\%}. Ini membuktikan bahwa \textit{framework translation-decomposition-search-resolve} dapat memanfaatkan kemampuan model untuk menghasilkan logika yang lebih tepat ketika diarahkan dengan benar.
		      \item Sebaliknya, pada model \textbf{Qwen2.5}, performa justru turun drastis dari 81.00\% (\textit{Naive}) menjadi 14.00\% (\textit{Aristotle}). Hal ini menunjukkan bahwa sistem \textit{Neuro-Symbolic} memiliki sifat \textit{brittle}, kesalahan kecil pada satu tahap akan merusak keseluruhan hasil, juga model yang tidak dapat mematuhi atau terlalu patuh pada format \textit{prompt template} tidak akan menghasilkan penalaran logis yang baik, berbeda dengan \textit{Naive Prompting} yang lebih toleran terhadap kesalahan parsial.
	      \end{itemize}
\end{enumerate}

Secara keseluruhan, penelitian ini menyimpulkan bahwa untuk aplikasi logika berbahasa Indonesia dengan sumber daya terbatas (model kecil \& terkuantisasi), model regional seperti SEA-LION yang dipadukan dengan arsitektur Neuro-Symbolic menawarkan solusi yang lebih lebih baik dibandingkan mengandalkan intuisi atau \textit{greedy reasoning} model semata.

\section{Saran}
\label{sec:saran}

Berdasarkan temuan dan keterbatasan penelitian ini, disarankan beberapa langkah pengembangan untuk penelitian selanjutnya:

\begin{enumerate} 
	\item \textbf{Penerapan Grammar-Constrained Decoding:}
	      Untuk mengatasi masalah \textit{parsability} (seperti yang dialami Qwen), penelitian masa depan sebaiknya mengintegrasikan penegakan tata bahasa (seperti format GBNF di \texttt{llama.cpp}) saat inferensi. Ini akan memaksa model, bahkan yang terkuantisasi, untuk menghasilkan output yang 100\% valid secara sintaks, seperti mengikuti format FOL, sehingga evaluasi dapat berfokus murni pada logika semantik.
	\item \textbf{Fine-Tuning Khusus (Instruction Tuning):}
	      Alih-alih hanya mengandalkan \textit{Prompt Engineering} (\textit{few-shot}), disarankan untuk melakukan \textit{Fine-Tuning} ringan (seperti LoRA) pada model Sahabat-AI atau SEA-LION menggunakan dataset pasangan (Teks Indonesia $\rightarrow$ FOL).
	      Hal ini akan meningkatkan konsistensi model dalam menangani struktur kalimat kompleks Bahasa Indonesia.
	\item \textbf{Peningkatan Kompleksitas Dataset:}
	      Penelitian ini hanya terbatas pada dataset ProntoQA (logika deduktif sintetik).
	      Pengujian selanjutnya perlu memperluas cakupan ke dataset, seperti menggunakan dataset yang dipakai pada penelitian Aristotle aslinya, yaitu ProofWriter \cite{tafjord2021proofwritergeneratingimplicationsproofs} dan LogicNLI \cite{tian-etal-2021-diagnosing}.
\end{enumerate}